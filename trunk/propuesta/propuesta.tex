\documentclass[a4paper,10pt]{article}
\usepackage[spanish,activeacute]{babel}


%opening
\title{Propuesta de tesis de licenciatura}

\begin{document}
\maketitle
%\begin{flushleft}
 \begin{description}
  \item[Alumno:] Pablo Hern'an Rodr'iguez Zivic
  \item[Director:] Lic. Carlos Diuk
  \item[Titulo:] Hacia la generaci'on algor'itmica de m'usica
  \item[Contenido:] Al escuchar una pieza musical, uno es capaz de extraer informaci'on acerca del 
	estilo, la m'etrica, y tal vez teniendo mayor conocimiento sobre teor'ia musical, cuestiones relacionadas 
	con armon'ia y contrapunto. Todo m'usico es influenciado por otros m'usicos por diferentes razones, 
	y esta capacidad de ser influenciado est'a estrechamente relacionada con la capacidad de extraer 
	informaci'on 'util para la composici'on a partir de escuchar lo que otros ya han hecho.
	Ahora, la pregunta es: >Es posible darle a una computadora un conjunto de piezas musicales y que esta sea 
	``influenciada'' por este conjunto para componer nueva m'usica?
    
    Es importante notar la diferencia que 
	tiene este problema al an'alogo aplicado a lenguaje natural. En el lenguaje natural hay reglas sem'anticas
	mucho mas duras que en la m'usica puesto que al fin y al cabo en esta 'ultima, lo importante es que 
	``suene bien''.
	

	Si bien existen varios m'etodos de generaci'on de m'usica, muy pocos utilizan t'ecnicas de aprendizaje 
	que automaticen el proceso, y permitan aprender nuevos estilos de composici'on con m'inima
	incorporaci'on de conocimiento previo. Un ejemplo de un trabajo que utiliza t'ecnicas de aprendizaje es 
	\cite{DaCo}. En 'el se trabaja sobre c'omo construir una especie de contrapunto conocida como contrapunto 
	de dos voces. Dentro de la teor'ia musical, el contrapunto es una rama que se dedica al estudio de la 
	interacci'on entre distintas voces dentro de una pieza musical. Un problema fundamental que se tiene 
	al componer una pieza utilizando un enfoque contrapunt'istico es que las reglas que esta teor'ia propone 
	premiten que una composici'on llegue a un ``callej'on sin salida'', haciendo que no se pueda continuar 
	con la construcci'on del tema respetando las reglas del contrapunto. David Cope propone en \cite{DaCo} 
	un algoritmo que aprende de las veces que tuvo que realizar backtracking al llegar a un ``callej'on sin 
	salida'', para no volver a pasar por esa situaci'on nuevamente. Si bien este trabajo es muy
	interesante, es muy limitado puesto que ataca s'olo a una especie en particular de contrapunto.
	 
	
	Dentro del contexto del an'alisis musical se ha estudiado bastante lo que respecta a extracci'on de 
	informaci'on de una pieza musical dada\cite{FXPAL}\cite{BeMe}\cite{FooJoCooMa}. 
	Para esto se suelen utilizar lo que se llaman \emph{matrices de similitud}.  Una matriz de similitud, 
	es una matriz que almacena en la posici'on $i, j$ cuanto se parecen el momento $i$ al momento $j$ 
	para un tema en cuesti'on con respecto a una m'etrica. 


	Por otro lado, se ha estudiado c'omo crear un acompa~namiento para una melod'ia dada\cite{MySong}. 
	Esto se logra entrenando un Hidden Markov Model con una serie de temas para los que se conoce su 
	partitura. El Hidden Markov Model se entrena de forma tal que este sepa qu'e acordes suelen acompa~nar 
	a qu'e melod'ias.  Es importante notar la diferencia de este trabajo con el que se propone aqu'i: 
	en \cite{MySong} no se crean nuevas melod'ias.


	Dentro del contexto de information retrieval, se ha estudiado c'omo indexar audio para soportar audio
	queries \cite{IoAlApYa}. En este trabajo en particular, atacan este problema a partir de codificar a cada 
	tema como un conjunto de ``features'' y luego, para realizar una b'usqueda, encontrar el tema que 
	se encuentra m'as cercano en este espacio de ``features''. 


	El objetivo de esta tesis es estudiar modelos de machine learning aplicados a resolver el problema que
	se plantea inicialmente: Dado un conjunto de temas, y ciertos guidelines generales sobre la estructura
	del tema, construir un tema influenciado en el conjunto dado. Para esto se propone estudiar modelos 
	principalmente jer'arquicos relacionados con Hidden Markov Models y Bayesian Networks.

 \end{description}
%\end{flushleft}

\begin{thebibliography}{99}
\bibitem[1]{FXPAL} Foote Jonathan; Visualizing Music and Audio using Self-Similarity
\bibitem[2]{BeMe} Meudic Benoit; Musical pattern extraction: from repetition to musical structure
\bibitem[3]{MySong} Ian Simon, Dan Morris, Sumit Basu; MySong: Automatic Accompaniment Generation for Vocal Melodies
\bibitem[3]{FooJoCooMa}Foote Jonathan,Cooper Matthew; Automatic Music Summarization via Similarity Analysis
\bibitem[4]{IoAlApYa}Ioannis Karydis, Alexandros Nanopoulos, Apostolos N. Papadopoulos, Yannis Manolopoulos; Audio Indexing for Efficient Music Information Retrieval
\bibitem[5]{DaCo} Cope David; A Musical Learning Algorithm


\end{thebibliography}

\end{document}
