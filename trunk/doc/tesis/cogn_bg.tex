\section{Background cognitivo y musical}
\label{sec_cogn_bg}
Principalmente, la m'usica tiene dos grandes componentes: la del \texttt{tiempo} y la de la 
\texttt{altura}. La dimensi\'on del tiempo se refiere a las duraci\'ones de los distintos eventos que 
ocurren en una pieza musical. Por otro lado, la altura se refiere a la percepci\'on de las distintas notas que ocurren 
en un tema. Es importante notar la diferencia entre la altura de una nota y la frecuencia de la misma; la altura refiere
a la percepci'on de una frecuencia.

Si bien estas dos dimensiones se presentan por separado, de ninguna manera son independientes. Ser'an abordadas de esta forma 
para facilitar una primera descripci'on. De ser necesario, posteriormente se elaborar'an los conceptos que refieran a las interacciones entre 
estas dos componentes. 

\subsection{La tensi\'on y lo esperable}
\label{subsec_tension}
\begin{itemize}
  \item 

\end{itemize}
\comment{aca voy a hablar sobre la relacion que hay entre la tension y lo esperable, definiendo como esperable a un evento en un cierto contexto, luego voy a distinguir entre
dos tipos de contextos: el horizontal y el vertical. El contexto horizontal tiene que ver con patrones en el tiempo, y el vertical con relaci'ones armonicas en un cierto
instante. Esto despues lo voy a usar para los modelos de la ritmica (solo usa contexto horizontal) y de la melodia (usa los dos contextos y arma un modelo conjunto)}


\subsection{El tiempo es racional: el concepto de \emph{beat}}
\comment{Esto no lo borro solo porque me costo escribirlo, pero creo que aca no pincha ni corta }\newline
En su libro, Lerdahl y Jackendoff(\cita) hacen una distinci'on entre dos estructuras que ocurren en simultaneidad en la m'usica:
La estructura \emph{m'etrica} y la estructura del \emph{agrupamiento}\footnote{\emph{meter} y \emph{grouping} en ingl'es}. 
Esta distinci'on en t'erminos generales coincide con clasificaciones de otros autores (\alert{citas muchas!}), se toma la version de Lerdahl y Jackendoff 
por ser esta m'as computacional.
La estructura del agrupamiento hace referencia a la organizaci'on de una pieza musical en unidades que pueden ser motivos, frases, secciones, etc. 
Cada una de estas unidades, es denominada por los autores \emph{grupo}. Asimismo, el oyente infiere una estructura regular de pulsos. 
Algunos pulsos son m'as fuertes que otros, determinando lo que los autores definen como la estructura m'etrica. 
Para ponerlo en concreto pensar la estructura como la forma en la que un director de orquesta mueve su batuta. 
En lo subsiguiente se utilizar'an los t'erminos pulso y \emph{beat} como sin'onimos intercambiables.

Ambas estructuras tienen una forma jer'arquica, en el sentido de que existen sub-estructuras a distintos niveles, y que las sub-estructuras de niveles superiores 
incluyen por completo a las de nivel inferior\footnote{Para una definici'on mas formal de la jerarqu'ia referirse a \cita}. De esta forma, una secci'on
de una pieza musical estar'a formada por una sucesi'on de frases. Estas frases s'olo pertenecer'an a esa secci'on, sin embargo, esto no quiere decir
que no se pueda repetir una frase en dos secciones distitnas, puesto que cada una pertenecer'a a una sola secci'on. 
Esto mismo ocurre con la estructura m'etrica; existen distintos niveles de beats dados por el tiempo que ocurre entre dos pulsos sucesivos. De esta forma, al 
ocurrir estos en tiempos regulares, s'olo hace falta saber cual es la distancia entre cualquier par de beats consecutivos para referirse a ese nivel. 

La organizaci'on jer'arquica de la m'usica no es una teor'ia solo postulada por Lerdahl y Jackendoff; Cooper y Meyer(\cita) plantean una teor'ia similar en
estos t'erminos, Kramer(\cita) si bien no habla directamente de una jerarqu'ia, propone que tienen un rol 
estructurante\footnote{\alert{tengo que leer m'as, as'i se mejor de lo que hablan estos dos muchachos}}. 

Un nivel de especial inter'es, es el denominado \emph{tactus}, que b'asicamente es el marcado por el director de orquesta al mover su batuta. 
El tactus tambi'en es la distancia entre los pulsos que el oyente marca cuando mueve el pie y est'a relacionado con el baile. 

Es importante notar que esta estructura es ambigua, en el sentido de que que muchas veces no existe una 'unica descomposici'on de una pieza musical 
en una estructura m'etrica y de agrupamiento.


\subsection{Acentuaci\'on}
Una car'acteristica de la m'usica, tambi'en compartida con el habla, es que un mismo evento no es percibido de la misma forma seg'un
el contexto en el que ocurre. Hay varios factores que afectan el contexto, y uno de ellos es la \texttt{acentuaci\'on}. 

Un evento musical es escuchado como acentuado si es enfatizado de alg\'una forma. Lerdahl y Jackendoff(\cita) distinguen tres tipos de 
acentos: los acentos fenomenales \footnote{\alert{se que no es esta la traducci\'on, como se dice en espa~nol?}}, estructurales y m'etricos.
Un acento fenomenal es cualquier evento que de 'enfasis o estress a un momento en la pieza musical. \alert{Que ejemplos doy? esto lo va a 
leer gente que \emph{no sabe} musica}. Los acentos estructurales son puntos de apoyo para finalizar una parte o una frase. 
Por 'ultimo, los acentos m'etricos son aquellos beats relativamente fuertes dentro del contexto m'etrico donde suceden.

Kramer(\cita) tambi'en reconoce tres tipos distintos de acentos, llamados \emph{m'etrico}, \emph{stress} y \emph{r'itmico}. Esta categorizaci'on
es equivalente a la dada por Lerdahl y Jackendoff.

Meyer(\cita) no elabora\footnote{checkear que sea realmente asi} una taxonom'ia de acentos. Su inter'es no est'a en saber \emph{qu'e} hace
que un pulso est'e acentuado sino \emph{c'omo} opera un pulso acentuado en terminos de estructura r'itmica. En estos t'erminos distingue
que los pulsos acentuados son de alguna forma el punto de foco donde los beats circundantes se agrupan. 

Es importante notar que este tipo de categorizaciones no son excluyentes, es decir, un beat puede tener m'as de un tipo de acento al mismo tiempo.
Para ejemplificar, dentro de la teor'ia de Lerdahl y Jackendoff se describe el proceso cognitivo por el cual se infieren los acentos m'etricos. Este
proceso consiste en que el oyente utiliza los acentos fenomenales y estructurales como pistas para extrapolar un patr'on regular de acentos m'etricos. 
Dentro de 'este contexto, no es la excepci'on, sino la regla que haya beats que son acentuados con un acento fenomenal y m'etrico al mismo tiempo.


