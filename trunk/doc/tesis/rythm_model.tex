\section{Modelando la m\'etrica}
\label{sec:metric_model}
\subsection{Fundamentos}
El acento m'etrico es una caracter'istica inherente a la m'usica tonal; cualquier pieza musical se enmarca en una estructura 
(posiblemente ambigua) de beats fuertes y d'ebiles. Si bien 'esta existe solamente en la mente del escucha, las reglas para inferirla son culturalmente conocidas 
y colocan al acento m'etrico como un elemento importante dentro del lenguaje musical. La estructura m'etrica ayuda a medir el tiempo permitiendo a un interprete 
reproducir y a un oyente reconocer un cierto conjunto de relaci'ones temporales. Esta estructura da contexto para interpretar la r'itmica,  
donde por r'itmica se refiere a un grupo de al menos dos eventos musicales en donde hay uno acentuado con respecto al resto \cite{CooperMeyer60}.
Es por esto que un patr'on temporal as'i descripto es interpretado de forma distinta seg'un el contexto m'etrico donde sea escuchado. 

Siguiendo la teor\'ia de Lerdahl y Jackendoff el acento m'etrico es una construcci'on que si bien es jer'arquica, lo niveles altos no son percibidos, puesto
que en esos niveles entran en juego los acentos estructurales y los acentos fenom'enicos. De esta forma se habla de la \emph{localidad} de los acentos m'etricos: 
s'olo se perciben a niveles bajos en la jerarqu'ia. 

La estructura m'etrica permite organizar los distintos puntos en el tiempo en clases de 
equivalencia de forma tal que dos puntos distintos en el tiempo que pertenezcan a la misma clase de equivalencia ser'an percibidos de forma similar\cite{Benjamin84}. 
De esta forma, es posible pensar al paso del tiempo como saltos entre estas clases de equivalencia.

Es importante resaltar que dentro de la definici'on de acento m'etrico, se encuentra la condici'on de que este debe ser regular, es decir, \emph{peri'odico}. 

Las propiedades de \emph{localidad}, \emph{periodicidad} y \emph{equivalencia} permiten realizar ciertas asunciones sobre las dependencias entre variables aleatorias 
a tener en cuenta para modelar la m'etrica. En primer lugar, la propiedad de localidad, hace que se pueda asumir que no existen dependencias fuertes entre eventos distantes 
en el tiempo, permitiendo desarrollar un modelo simple en esos t'erminos.  A su vez, la propiedad de equivalencia, permitir'a modelar estas clases de equivalencias, 
definiendo el paso del tiempo como saltos entre ellas. Por 'ultimo, la propiedad de periodicidad, establece que el fen'omeno no s'olo es local, 
sino que se vuelve a comenzar en per'iodos fijos en el tiempo.

\subsection{El modelo}
\label{sec:rythm_model}
El hecho de que exista un per'iodo, y que se vaya a utilizar para la construcci'on del modelo hace que sea necesario estimarlo. Por ahora, as'umase dado un valor 
$p$ que determina el per'iodo como un m'ultiplo entero de un beat. M'as adelante se exhibir'an distintas aproximaci'ones para inferir el valor $p$. Adem'as
de esto, por ahora as'umase tambi'en que la pieza no tiene notas sonando en simultaneo.

La asunci'on de que no hay notas sonando en simultaneo permite representar a la partitura como una sucesi'on de onsets, y utilizando esto, se define 
de funci'on de clases de equivalencia $c$ de la siguiente forma:

\begin{definition}
Dada una nota $n$, se define su clase de equivalencia $c$ como $$c(n) = onset(n)\mod p$$
\end{definition}

Esta definici'on pretende capturar cierta esencia de la noci'on de clase de equivalencia entre acentos m'etricos definida por \cite{Benjamin84}. Observar que asumendo 
una elecci'on de $p$ adecuada, todos los posibles acentos m'etricos son representados, sin embargo, distintos valores de la funci'on $c$ podr'ian recibir el mismo acento m'etrico 

P'ongase como ejemplo el de la partitura de Figura \ref{fig:simple_measure}. La sucesi'on de onsets que ocurren en ella son 
$$0, \frac{3}{4}, 1, 2, 2+\frac{3}{4}, 3$$ 
A modo de ilustraci'on, se elije $p=4$, de esta forma, la sucesi'on de clases de equivalencia ser'a igual a la sucesi'on de onsets, puesto que el mayor de los onsets es menor 
estricto al per'iodo. Dado que el comp'as exhibido en la Figura \ref{fig:simple_measure} es un comp'as de $\frac{4}{4}$, tanto el primer beat como el tercero reciben la misma 
acentuaci'on m'etrica, sin embargo, la funci'on $c$ les asigna distintas clases de equivalencia.


\begin{imagen}
    \file{images/simple_measure.png}
    \labelname{fig:simple_measure}
    \desc{Un comp'as que ejemplifica la no inyectividad de la funci'on $c$}
\end{imagen}

A partir de esta funci'on, es posible traducir una l'inea mel'odica $n_1,\cdots,n_k$ en una sucesi'on de clases de equivalencia $c(n_1),\cdots,c(n_k)$. 

Adem'as, suponiendo que el evento (nota o silencio) que m'as dura, tiene una duraci'on menor o igual al per'iodo; formalmente 
$$\max_{0\leq j \leq k-1}o_{j+1}-o_j \leq p$$ 

es posible reconstruir la secuencia de onsets original mediante la siguiente transformaci'on:
$$o_i=c(n_i) + \#reset(i)\times p$$ siendo $\#reset(i)=\#\{c(n_j) \leq c(n_{j+1}), j+1 \leq i\}$. En lo sucesivo, se referir'a a esta propiedad como la \emph{pseudo-biyectividad} 
entre la sucesi'on de clases de equivalencia y la sucesi'on de onsets.

De esta forma, se puede construir un modelo utilizando la funci'on $c$ como mecanismo de generalizaci'on y la generaci'on de onsets es un'ivoca a partir de la secuencia de 
clases de equivalencia. Es por eso que se propone que un modelo generativo que aprenda el lenguaje $c_i=c(n_i)$ ser'ia entonces capaz de generar una sucesi'on de onsets 
que sean interpretados m'etricamente de la misma forma que la fuente de entrenamiento. 

El principio de localidad permite realizar la asunci'on de que no existen dependencias a largo plazo en el acento m'etrico, de esta forma se propone aprender este lenguaje mediante
una cadena de Markov de orden 1. 

Formalmente, el espacio de estados ser'a el conjunto de clases de equivalencia $S=\{c(n_i), i\leq k \}$, y la matriz de transici'on estar'a definida de forma standard:
$$p(c_j|c_i) = \frac{n(c_i, c_j)}{n(c_i)}$$
siendo $n(c_i, c_j)$ la cantidad de veces que se observ'o la clase de equivalencia $c_j$ luego de la clase de equivalencia $c_i$, y $n(c_i)$ es sencillamente la cantidad de 
veces que se observ'o la clase de equivalencia $c_i$.

\subsection{Ejemplo}
La figura \ref{fig:reggae_rhythm}  muestra una partitura sencilla con dos c'elulas r'itmicas.
\begin{imagen}
    \file{images/reggae.png}
    \labelname{fig:reggae_rhythm}
    \desc{Dos compases para entrenar el modelo}
\end{imagen}

Suponiendo que el valor elegido para el per'iodo $p$ es el de $4$ negras, entonces, la partitura quedar'ia segmentada en sus dos compases. 
De esta forma, la secuencia de onsets medidos en multiplos de una negra ser'a para el primer comp'as
$$0, \frac{3}{4}, 1, 2, 2+\frac{3}{4}, 3$$
y para el segundo comp'as 
$$4, 4+\frac{1}{3}, 4+\frac{2}{3}, 5, 5+\frac{1}{3}, 5+\frac{2}{3}, 6, 6+\frac{3}{4}, 7$$

Por lo tanto, cocientando los onsets m'odulo el per'iodo y estimando las transiciones resulta la cadena de Markov de la figura \ref{fig:reggae_rhythm_markov}.
\begin{imagen}
    \file{images/rhythm_markov_example.png}
    \labelname{fig:reggae_rhythm_markov}
    \desc{El modelo generado a partir de la r'itmica de la Figura \ref{fig:reggae_rhythm_markov}}
\end{imagen}

No se han dibujado las probabilidades en las transiciones puesto que no tienen sentido en este ejemplo por ser muy chico. Sin embargo, para ilustrar la cuenta:
$$p(\frac{1}{3}|0) = \frac{n(0, \frac{1}{3})}{n(0)} = \frac{1}{2}$$

\subsection{El proceso generativo}
Dada la propiedad de \emph{pseudo-biyectividad} entre la sucesi'on de clases de equivalencia y la suceci'on de onsets, el proceso generativo quedar'a esquematizado de la siguiente 
forma:

\begin{enumerate}
  \item Sea $e$ el estado actual en la cadena de Markov. 
  \item Sea \emph{offset} la suma de los onsets generados hasta el momento. 
  \item Elejir el estado $f$ seg'un la matriz de transiciones
  \begin{enumerate}
    \item Si $f > e$, entonces sea $onset=$ \emph{offset} $+ f-e$
    \item Si $f \leq e$, entonces sea $onset=$ \emph{offset} $+ f + p - e$
  \end{enumerate}
  \item Generar un onset con el valor de $onset$
  \item Actualizar \emph{offset} $= onset$
\end{enumerate}

El proceso generativo comienza en el estado denominado como $0$, y con la variable \emph{offset} inicializada en $0$.

\subsection{Sobre la generaci\'on de r\'itmicas no vistas previamente}
Una propiedad deseable que cualquier modelo propuesto en este trabajo, en particular el reci'en descripto, es que sea capaz de generar nuevo material, distinto al de entrenamiento
pero relacionado. En este caso, el nuevo material ser'ia r'itmica \emph{nueva}, y la relaci'on con la r'itmica de entrenamiento, es que el acento m'etrico inferido por un oyente 
sea el mismo al que se infiere de la r'itmica utilizada para el entrenamiento. 

Retomando el ejemplo definido en la figura \ref{fig:reggae_rhythm}, notar que el modelo planteado es capaz de generar la r'itmica de la figura \ref{fig:rhythm_markov_generated}
que no es ninguno de los dos ejemplos de entrenamiento.

\begin{imagen}
    \file{images/reggae_generated.png}
    \labelname{fig:rhythm_markov_generated}
    \desc{Una posible r'itmica generada por el modelo}
\end{imagen}

%\comment{relacionar esto con lo de contexto horizontal definido en \ref{subsec_tension}}

\subsection{Infiriendo el valor del per\'iodo}
Hasta ahora se dio por dado el valor $p$, sin embargo es un valor cr'itico para el modelo, puesto que forma una parte importante del mecanismo de generalizaci'on que se 
utiliza. M'as all'a de como sea inferido, este valor deber'ia representar el nivel en la jerarqu'ia m'etrica a partir del cual, en los niveles superiores el acento m'etrico 
pasa desapercibido del resto de los acentos. Se entiende que este valor es subjetivo, y que no existe un l'imite claro a partir del cual se puede afirmar esto.

Dado que la entrada al algoritmo, es de hecho, una partitura, una primera aproximaci'on para inferir el valor $p$, es utilizar la informaci'on del comp'as. 
Esta aproximaci'on tiene la ventaja de que no es necesario realizar ning'un c'omputo, sin embargo, el comp'as escrito en la partitura no necesariamente refleja la jerarqu'ia
m'etrica inferida por un oyente al escuchar esa partitura puesto que el comp'as en la partitura escrita tambi'en cumple un rol organizativo que facilite la lectura. 

Una segunda aproximaci'on consistir'ia en dise~nar un algoritmo que infiera la jerarqu'ia de acentos m'etricos de la partitura. Esto se ha realizado con anterioridad por
\cite{Temperley2001}. Dentro de este software se encuentran implementadas distintas reglas de preferencia definidas por Lerdahl y Jackendoff y 
en particular las relacionadas con el acento m'etrico. El problema al inferir la jerarqu'ia m'etrica, es luego decidir cual ser'a el nivel elegido para utilizar como per'iodo.


\subsection{Traducci'on cuando hay notas en simult'aneo}
En el transcurso de esta secci'on se realiz'o la asunci'on de que no existen notas sonando en simult'aneo en la pieza de entrenamiento, sin embargo esta es una asunci'on poco
realista.


\begin{imagen}
    \file{images/multi_rythm.png}
    \labelname{fig:multi_rythm}
    \desc{Una partitura con notas en simult'aneo}
\end{imagen}

En la figura \ref{fig:multi_rythm} obs'ervese que no s'olo existen notas que suenan en simult'aneo, los acordes de Do mayor y de Fa mayor, 
sino que durante la nota C hay notas que comienzan luego que ya esta sonando la nota C. Esto requiere que se deba redefinir la traducci'on
a estados de la cadena de markov, puesto que la traducci'on no queda un'ivocamente determinada en este escenario.
Si se realizara una traducci'on como se defini'o en \ref{sec:rythm_model}, el resultado de esta depender'ia del orden en que se pongan
las notas de los acordes. Se exhibe a continuaci'on una posible traducci'on:

$$0, 1, 2, 2, 3, 3$$

Notar que hay dos 2 y dos 3, puesto que el C suena desde la negra 1. Esta traducci'on es irreal, puesto que tener el onset 2 seguido del onset 
en realidad quiere decir que se observ'o una nota de duraci'on $p$. Deber'ia ser indistinto a la traducci'on el hecho de que existan notas 
con las caracter'isticas mensionadas antes. De esta forma, se propone, en lugar de generar una sola secuencia de entrenamiento para la cadena
de markov, generar una sucesi'on de secuencias de entrenamiento. Dentro de esta sucesi'on, cada elemento ser'a una secuencia de entrenamiento
que correspondr'a a la observaci'on de una nota con su onset y su \emph{offset}, siendo este 'ultimo el onset m'as la duraci'on. Es decir, 

\begin{definition}
Dada una nota $n$, se define su secuencia de entrenamiento como un par ordenado $$(onset(n) \mod p, \text{\emph{offset}}(n)\mod p)$$
\end{definition}

De esta forma, asumiendo $p=4$, la traducci'on ahora ser'a:

\begin{figure}
\begin{center}
\begin{tabular}{c | c} 
onset & offset \\
\hline
0 & 1 \\
1 & 0 \\
2 & 3 \\
2 & 3 \\
3 & 0 \\
3 & 0 \\
\end{tabular}
\caption{ Traducci'on correcta de la partitura de la figura \ref{fig:multi_rythm}}
\label{fig:multi_rythm_translation}

\end{center}
\end{figure}

Observar que de graficar la cadena de Markov, esta quedar'ia disconexa en este caso. Esto ocurre porque el ejemplo es muy chico, en piezas musicales mayores esto deja de ocurrir.
