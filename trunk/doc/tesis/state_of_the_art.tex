\section{Aportes de este trabajo}
Se considera como aporte principal de este trabajo la metodolog'ia de validaci'on propuesta: 

Contando con modelos generativos aislados para distintos atributos musicales, es posible validar los fundamentos de cada modelo
realizando una serie de composiciones en donde se ve afectando el funcionamiento del modelo correspondiente.

Para poder alcanzar este objetivo, fue necesario construir modelos ge\-nerativos con base en teor'ias psicol'ogicas para los distintos atributos que 
caracterizan a una melod'ia. 

De este modo, la contribuci'on de este trabajo se sintetiza en la siguiente lista de modelos (muchos de los t'erminos relacionados con teor'ia musical se definir'an
m'as adelante):

\begin{itemize}
 \item Modelo para la r\'itmica: Basado en 3 propiedades perceptuales del acento m'etrico, fue posible construir un modelo sencillo para la r'itmica
 que cumpla con la propiedad de que la r'itmica generada har'a que un oyente infiera la misma estructura m'etrica que la que el 
 oyente inferir'ia en la obra original.

 \item Modelo para los contornos mel'odicos: Basado en la teor'ia de Implicaci'on-Realizaci'on de \cite{Narmour90}, se propone un modelo simple que permite cuantificar 
 el grado de implicaci'on que genera un intervalo mel'odico sobre el siguiente.

 \item Modelo para contextos arm'onicos: Basado en los trabajos de \cite{Krumhansl90} y \cite{Lerdahl2001} se propone un modelo basado en un marco bayesiano
 para cuantificar la estabilidad de una nota en funci'on de la tonalidad inferida del tema y el acorde que gobierna la sonoridad del momento.

 \item Modelo para frases mel'odicas: Utilizando los modelos anteriores, y basado en algunas definici'ones de en qu'e consiste una 
 frase musical, se construye un modelo que organize la melod'ia en t'erminos de unidades discursivas.

 \item Modelos para generar relaciones mot'ivicas: Se propone utilizar el modelo de los Restaurantes Chinos \citep{Teh2007} para generar repetici'on 
 de partes, de forma tal que la cantidad de repeticiones de una parte no sea excesiva ni nula.

\end{itemize}


A modo de resumen, en la figura \ref{fig:arquitectura} se exhibe la arquitectura de los modelos utilizados y en qu'e cap'itulo se aborda cada uno.

\begin{imagen}
    \file{images/arq.png}
    \labelname{fig:arquitectura}
    \desc{Descripci'on gr'afica de la arquitectura utilizada}
    \width{10.5cm}
    \position{!h}
\end{imagen}

\begin{itemize}
 \item Pitch Profile: Construye el pitch profile definido en la secci'on \ref{sec:pitch_profile}.
 \item Chord Detection: Aplica una heur'istica para detecci'on de acordes.
 \item Contour Patterns: Calcula el modelo definido en la secci'on \ref{sec:contour_model}.
 \item Notes Distribution: A partir del Pitch Profile y del acorde actual construye la distribuci'on de notas que aplica al contexto actual seg'un lo definido en la secci'on \ref{sec:harmonic_context_model}.
 \item Phrase Repetition: Utilizando el Restaurant Chino correspondiente al acorde actual, se elije un identificador para la parte actual como 
 se defini'o en la secci'on \ref{sec:crp_model}.
 \item Phrase Rhythm: Utilizando la duraci'on determinada por el acorde actual y el identificador determinado por la etapa de Phrase Repetition se 
 genera una frase r'itmica como se defini'o en la secci'on \ref{sec:rythm_model}.
 \item Pitch Phrase: Utilizando la r'itmica generada por la etapa de Phrase Rhythm, se asignan notas utilizando como contexto arm'onico el determinado 
 por Notes Distribution y utilizando como restricci'ones para el contorno mel'odico las determinadas por Contour Patterns utilizando 
 el algoritmo definido en la secci'on \ref{sec:phrase_model}.
\end{itemize}


\section{Estado del arte}
\label{sec:state_art}
A continuaci'on se har'a un breve resumen sobre los distintos enfoques y objetivos relacionados con el an'alisis musical mediante
t'ecnicas computacionales. Las aplicaciones de este tipo de trabajos son bastante variadas, tales como herramientas para 
composici'on musical asistida, acompa~namiento autom'atico, indexaci'on de m'usica, sistemas de recomendaci'on musical e 
investigaci'on en psicolog'ia de la m'usica, donde se sit'ua este trabajo.

De acuerdo al enfoque que se tome, puede variar la representaci'on musical utilizada entre la se~nal sonora cruda, 
y una partitura\footnote{o midi, que es equivalente a niveles pr'acticos }.
En este trabajo se asumir'a una representaci'on simb'olica de la m'usica para poder desarrollar en profundidad las teor'ias cognitivas de las 
que luego se hablar'a. A continuaci'on se nombran algunos trabajos que se enmarcan de la misma forma en t'erminos de la representaci'on musical.

\cite{PaieThesis} propone un modelo generativo para l'ineas mel'odicas, 
patrones r'itmicos y armonizaci'ones, basado en algunos de los principios que se utilizar'an en este trabajo. 
Sin embargo, no es el objetivo del autor poner a prueba teor'ias cognitivas de la m'usica; sino desarrollar un modelo generativo,
y por lo tanto predictivo, basado en una representaci'on simb'olica de la m'usica de forma tal que se pueda utilizar el conocimiento generado por estos
modelos para mejorar la calidad de algoritmos que trabajan con se~nales sonoras. Este trabajo aborda el problema de la estabilidad tonal a partir
de entrenar una cadena de Markov que tiene estados ocultos que representan los acordes, aspecto que puede ser abordado utilizando las teor'ias 
de \cite{Krumhansl90} y \cite{Lerdahl2001}. De la misma manera, el modelo de la r'itmica utiliza una definici'on incorrecta de acento m'etrico, asumiendo
que tiene caracter'isticas globales. Si bien estos modelos podr'ian funcionar en el marco en el que el autor los plantea, en este marco no son de utilidad.

\cite{Shih-Chuan} proponen un software que mediante la utilizaci'on de t'ecnicas de data mining componga m'usica. Estas t'ecnicas son aplicadas
para descubrir patrones en la composici'on musical en su estructura r'itmica y mel'odica.
Si bien estos modelos no son de inter'es para este trabajo por no basarse directamente en fundamentos cognitivos, 
la arquitectura que presentan \cite{Shih-Chuan} sirvi'o como punto de partida para construir la presentada en este trabajo. 

Otro software existente dise~nado para componer m'usica es el Melisma Music Generator
\footnote{Disponible en http://www.link.cs.cmu.edu/melody-generator/} basado en \cite{Temperley2004}. 
Este software se encuentra disponible para escuchar online sus composiciones, presentando una interfaz web donde es posible ingresarle distintos
valores a sus par'ametros y escuchar como estos impactan en la composici'on final. 
El mayor problema 
que tiene es que no se puede entrenar directamente su modelo con una partitura, y por m'as que se intentara realizar esto construyendo un software 
que estime valores posibles para sus par'ametros, no habr'ia forma de determinarle ciertos par'ametros que son utilizados en este trabajo, 
como la sucesi'on de contextos arm'onicos.


%David Cope, en trabaja con un software, que mediante reglas, sea capaz de reproducir el estilo de una pieza musical dada. Si bien 
%la construcci'on ha sido exitosa en algunos casos, generando composiciones que realmente respetan el estilo de la pieza original, gran
%parte de las reglas utilizadas no tienen sustento cognitivo.
%

\cite{Cambouropoulos98} propone una teor'ia general para el an'alisis musical. 
Su trabajo consiste en una teor'ia computacional que permitir'ia obtener una descripci'on de la estructura de una cierta obra musical. Si bien sus 
objetivos son distintos a los que se proponen aqu'i, no distan tanto: 
\begin{quote}
[\ldots]Teor'ias musicales permiten formular hip'otesis y modelos que pueden ser implementados
como programas para luego ser evaluados, y, de forma inversa, resultados de la aplicaci'on de
estos programas podr'ia forzar la re-examinaci'on y ajuste de las teor'ias iniciales. \footnote{[\ldots]Musical theories allow the formulation of hypotheses and models which can be implemented
as computer programs and then evaluated, and, conversely, results from the application of the
computer programs may force the re-examination and adjustment of the initial theories. [\ldots]
} [\ldots]

\end{quote} 
A lo largo de este trabajo no se utiliz'o esta teor'ia y se opt'o por teor'ias que han sido m'as ampliamente discutidas tanto en el marco de la musicolog'ia
como en el marco de la psicolog'ia de la m'usica.

%
%En \cite{Simon_mysong:automatic} mediante el an'alisis de caracter'isticas de la se~nal sonora correspondientes a una melod'ia 
%cantada, y mediante entrenar un Hidden Markov Model para la probabilidad de un acorde, dado que se observa una cierta nota cantada,
%se construy'o un sistema capaz de armonizar una melod'ia. Un sistema comercial que realiza esto mismo es Band-in-a-box,
%pero dado que no existen \red{revisar} publicaciones respecto a c'omo fue construido, no se puede hacer m'as que nombrarlo.
%
%Dentro de la rama de sistemas para indexar m'usica se situan trabajos como \cite{StructureAnalysis1}. En 
%\cite{StructureAnalysis1} se propone un m'etodo para analizar la estructura de una pieza musical a partir de su se~nal sonora. Esto lo logran extrayendo vectores 12-dimencionales
%de cada momento del tema, en donde cada componente del vector muestra la intencidad relativa de cada altura en ese momento, y a partir
%de estos vectores y una noci'on de distancia construyen matrices de similitud que permiten detectar los acordes que aparecen
%en el tema. Siguiendo con este 

El resto de la tesis se organiza de la siguiente forma: En el cap'itulo 2 se cubren superficialmente los conceptos b'asicos que el lector debe manejar para entender
este trabajo. Se alienta a los lectores no iniciados en alguna de las tem'aticas abordadas profundizar tales conceptos en la literatura referida. 
En el cap'itulo 3 se exhibe modelos que capturan dependencias a nivel local en la m'usica tonal. 
En el cap'itulo 4 se muestra modelos de 'indole jer'arquico y por 'ultimo en el cap'itulo 5 se reportan los experimentos realizados junto con 
las conclusiones del trabajo.
