\section{Modelando l\'ineas mel\'odicas}
En la secci'on anterior se realiz'o un an'alisis respectivo a la sucesi'on de duraciones que determina una cierta partitura, ignorando la 
suceci'on de alturas. En esta secci'on se proponen una serie de modelos para abarcar finalmente poder generar l'ineas mel'odicas.

\subsection{Contextos}
Las caracter'isticas definidas hasta ahora representan en gran parte las restricciones que deber'ian tenerse en cuenta en la construcci'on 
de l'ineas mel'odicas respecto a una cierta pieza musical, sin embargo, ninguna de ellas habla de propiedades de la l'inea mel'odica por si s'ola.

En 1960, Leonard Meyer elabor'o una teor'ia acerca de la expectativa en la m'usica aplicando principios gest'alticos de la psicolog'ia. La 
psicolog'ia de la Gestalt define principios que pretenden capturar la forma en que la mente configura los elementos que llegan a ella a trav'es 
de la percepci'on o de la memoria. Por ejemplo, la ley de cierre establece que nuestra mente a~nade los elementos faltantes a para completar una 
figura. De esta forma, en la figura \ref{fig:ley_cierre} se puede ver un c'irculo y un rectangulo, si bien en la imagen solo hay partes.

\begin{imagen}
    \file{images/Gestalt_ley_de_cierre.png}
    \labelname{fig:ley_cierre}
    \desc{Ejemplo de la ley de cierre. Si bien en esta imagen no hay un circulo ni un rectangulo, nuestra mente lo completa. \alert{>como pongo esto al costado?}}
    \width{6cm}
\end{imagen}

Luego del trabajo de Meyer, Eugene Narmour en (\cita) cuantifica estas reglas en t'erminos de intervalos para construir una teor'ia de 
la expectativa de los contornos mel'odicos. 

En lo que sigue, se explican con mayor detalle las teor'ias de Narmour y Lerdhal, para luego detallar el modelo de las l'ineas mel'odicas.


\footnote{definir pitch class y nota y equivalencia entre notas de diferente octava en las secciones de background} 

\subsection{La teor\'ia de la Implicaci\'on-Realizaci'on}
Como se mencion'o en la introducci'on de este cap'itulo, Eugene Narmour, basado en la teor'ia de la expectativa musical de Leonard Meyer, propuso una forma para cuantificar
el grado de expectativa sobre el \emph{contorno melod'ico}. El contorno mel'odico est'a conformado de la sucesi'on de intervalos que ocurren en una melod'ia. Por ejemplo, 
en la figura \ref{fig:simple_melody} se exhibe una melod'ia en donde se tocan las notas Do, Re, Do, Fa\#. 

\begin{imagen}
    \file{images/melody.png}
    \labelname{fig:simple_melody}
    \desc{Melod'ia simple de ejemplo}
    \width{11cm}
\end{imagen}

El contorno mel'odico de la figura \ref{fig:simple_melody} ser'a entonces \IM{2}, \IM{-2}, \IM{6}.
Notar que esta transformaci'on no es biyectiva, puesto que la melod'ia Re, Mi, Re, Sol\# tiene el mismo contorno.

\alert{este parrafo esta enterito sujeto a revision =D}

El modelo de la Implicaci'on-Realizaci'on(I-R) toma de (\cita) que el sistema cognitivo se encuentra organizado jer'arquicamente. En esta jerarqu'ia, los sistemas 
de percepci'on m'as simples, como la vista, se encuentran al fondo y los sistemas de percepci'on mas abstractos o complejos como la memoria o la 
resoluci'on de problemas se encuentran en el tope. De esta forma se distinguen dos tipos de procesos expectaci'on denominados procesos \emph{bottom-up} 
y \emph{top-down}. Los procesos bottom-up son aquellos procesos cognitivos en donde se parte de una informaci'on ubicada en los niveles bajos de la jerarqu'ia y se la 
elabora llev'andola a los niveles altos, mientras que los procesos top-down lo hacen en el orden inverso. 

Narmour propone que los procesos que regulan la expectaci'on mel'odica son en mayor medida de tipo bottom-up, es decir, parten de informaci'on puramente sensorial. 
Luego clasifica a los intervalos en dos tipos: \emph{implicativo} y \emph{realizado}. Los intervalos implicativos son aquellos que no promueven una sensaci'on de cierre, 
y por lo tanto generan implicaciones mel'odicas. Los intervalos realizados, como es de esperarse, promueven una sensaci'on de cierre, pero no necesariamente satisface
la implicaci'on generada por el intervalo implicativo. 
 
Teniendo entonces el contorno de una melod'ia, Narmour define ciertas caracter'isticas relacionadas con la psicolog'ia de la gestalt. A continuaci'on 
se enumeran los cinco principios de esta teor'ia, refiri'endose por intervalos peque~nos, a intervalos de valor absoluto menor o igual a 5 semi tonos, y por intervalos
grandes a aquellos que sean mayores o iguales que 7 semitonos, dejando al intervalo de 6 semitonos fuera de la clasificaci'on.
\begin{enumerate}
 \item Direcci'on registral: intervalos peque~nos implican continuaci'ones en la misma direcci'on mel'odica, mientras que intervalos grandes implican un cambio de direcci'on
 \item Diferencia interv'alica: intervalos peque~nos implican otros de tama~no similar y que intervalos grandes implican intervalos m'as peque~nos. 
 \item Retorno registral: se cumple cuando la segunda nota del intervalo realizado es id'entica o similar a la primera del intervalo implicativo.
 \item Proximidad: el tama~no del intervalo relizado sera peque~no.
 \item Cierre: se cumple cuando hay un cambio de direcci'on, un movimiento hacia un intervalo m'as peque~no, o ambas situaciones a la vez (\alert{reescribir})

\end{enumerate}

\subsection{Contexto vertical}
\comment{Yo aca uso un modelo que defini a dedo y que esta todavia sujeto a cambio, me gustaria discutir esto con Favio a ver si esta de acuerdo con lo que estoy haciendo. De todas formas me gustar'ia armar una discuci'on aca. Como contexto vertical, tambien se puede tener en cuenta la tonalidad (en menor medida que el acorde). Me gustaria ver si krumnhansl en su libro cognitive fundations of musical pitch me podria dar algo sobre como llenar esta parte o como hacer un modelo mas interesante (no se si quiero hacer un modelo mas interesante de todas formas =p)}

\subsection{El modelo}

\subsubsection{La cadena de Narmour}
\comment{aca la idea es explicar la cadena de markov de los intervalos de narmour}

\subsection{El proceso generativo}
\comment{Aca tengo que explicar el problema de mezclar estos dos modelos no es trivial, porque se te podria trabar el proceso porque llega a un callejon sin salida, y muestro
como deber'ia ser el proceso compuesto posta}
