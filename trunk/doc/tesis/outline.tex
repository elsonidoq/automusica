\section{Propiedades deseables de los modelos}
\mycomment{Me gustaria en esta seccion tratar de separar ciertas ideas que tengo que aplican a todos los modelos en general, y que justifican ciertas deciciones que tom'e al elejir los modelos.}
%En las siguientes secciones se presentan modelos que pretenden capturar ciertas caracter\'isticas del fen'omeno musical. 
%Si bien hay un grado de arbitrariedad inevitable al elejir los modelos, hay ciertos criterios comunes en la construcci'on de cada uno de ellos a aclarar previamente.
%
%Teniendo en mente el objetivo de este trabajo, la primer propiedad deseable que se puede pensar con respecto a los modelos, es que estos 
%permitan efectivamente \emph{producir} m'usica, es decir, tienen que ser capaces de generar m'usica. Se enfatiza adem'as el t'ermino ``producir'' para dar lugar a 
%la segunda propiedad deseable: la m'usica resultante de estos modelos debe ser de alguna forma nueva; distinta de la original. Es decir, cada uno de los modelos propuestos
%deber'a ser capaz de generalizar en pos de generar cosas nuevas.
%
%Por 'ultimo, la tercera propiedad, y no por ello menos importante es la elecci'on de qu'e caracter'istica se querr'a modelar y su correlato
%con la escucha musical.
