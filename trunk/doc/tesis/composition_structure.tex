\section{Modelando la estructura de la composic\'on}
\comment{la idea aca es explicar lo dificil que es modelar las elaboraciones motivicas, y que nadie logr'o hacerlo exitosamente, mucho menos yo =). Si bien mi modelo
no va a capturar exactamente lo que es una elaboraci'on mot'ivica, pretende tener ciertas propiedades similares}

\subsection{Restaurantes chinos}
El modelo de los restaurantes chinos (\emph{Chinese restaurants} \cita) es un modelo originado en la rama de estad\'istica bayesiana no param'etrica.
Este modelo y otros similares, como el Buffet Indio (\emph{Indian Buffet} \cita), han ganado mucha popularidad debido a que el proceso que generan
tiene una distribuci'on que permite estimar facilmente cual es la distribuci'on \texttt{a-posteriori} del proceso a partir de una distribuci'on 
\texttt{a-priori} (o creencia previa) y de un conjunto de observaciones.

Dado que el uso que se le dar'a a este modelo es puramente generativo, no se har'a incapi'e en los mecanimos necesarios para realizar inferencia\footnote{El lector interesado en profundizar en este tema refierase a \cita.}.
Concretamente, el modelo de los restaurantes chinos consiste en un restaurant con una cantidad infinita numerable de mesas\footnote{Escribir algo gracioso respecto al nombre}. 
En cada mesa, a su vez, pueden sentarse una cantidad no acotada de clientes. Inicialmente todas las mesas se encuentran vac'ias, 
y de a uno por vez empiezan a llegar los clientes. Cada cliente puede sentarse o bien en una mesa vac'ia o bien elejir una mesa de las ya ocupadas. 
Esto lo hace con la siguiente distribuci'on:
\begin{align}
P(\text{sentarse en mesa vacia}) =&\; n/(N + n)\\
P(\text{sentare en mesa } i) =&\; N(i)/(N + n)
\end{align}

Siendo $N$ es la cantidad total de clientes en el restaurant, $N(i)$ es la cantidad de clientes en la mesa $i$, y $n$ es un 
parametro que determina la \emph{concentraci'on} de clientes en cada mesa; se puede demostrar si $m$ es la cantidad de mesas ocupadas, entonces
$E(m|N) \in O(n\times log(N))$ 

\subsection{Las mesas son las partes}
\comment{Aca explico como por contexto armonico tengo muchos pedacitos de temas, y usando un restoran chino por contexto armonico puedo simular una cierta elaboraci'on motivica, que
no siempre anda bien (tengo que encontrar efectivamente casos en dodne no anda bien, pero recuerdo que los habia encontrado}

\subsection{Trabajo a futuro: modelos derivados}
\comment{aca me gustaria contar bastantes ideas que tengo sobre como a partir de un modelo de markov y un camino no simple (que pasa mas de una vez por algun nodo) en ese modelo,
construir un nuevo modelo donde ese camino sea lo mas probable, pero se permitan ``variaciones''}
