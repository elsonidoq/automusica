\section{Experimentos}
En esta secci'on se presentan una serie de experimentos que se realizaron con sujetos. Sin embargo, se considera
importante dar un panorama sobre como fueron implementados en la practica los algoritmos descriptos en los capitulos
anteriores, puesto que es fundamental ser capaz de controlar las variables para estar seguro que se esta poniendo
a prueba lo que se cree que se esta poniendo a prueba. De esta forma en lo subsiguiente se har'a una peque~na 
explicaci'on sobre la arquitectura utilizada, y como esta permite llevar a cabo experimentos confiables.

\subsection{Agentes independientes}
El primer concepto en el que se basa la arquitectura, es en el de \emph{agente independiente}. 
Cada algoritmo recibe una entrada y produce una salida. Los algoritmos se afectan entre s'i a partir de las 
interacciones dadas por la entrada/salida que producen, sin embargo, la implementaci'on interna de cada uno
no debe afectar. 

Es por esto que un efecto que es necesario aislar es la cantidad de consultas que hace cada algoritmo al
generador de n'umeros aleatorios; dado que los numeros aleatorios en la computadora no son mas que una secuencia
de numeros con propiedades estad'isticas que los hacen ver aleatorios, si un algoritmo consulta un n'umero m'as,
eso querr'a decir que otro algoritmo no podr'a consultarlo. De esta forma, cada agente deber'a contar con su propio
generador de n'umeros aleatorios.

Siendo adem'as que se desea poder intercambiar implementaciones de algoritmos, y que todos consulten 

Se hace incapi'e en esto 'ultimo puesto que distintas implementaciones de un algoritmo
podr'ian consultar 
