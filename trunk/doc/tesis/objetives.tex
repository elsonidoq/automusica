\section{Objetivos y alcance}
\comment{ya teniendo el background cognitivo y matematico (creo que el matematico no es necesario, pero queda feo que esten separadas las secciones de background) 
puedo definir mas precisamente cual es el objetivo y el alcance de esta tesis}
\newline

\comment{\Large{Objetivo mal y pronto: Tomar teorias cognitivas de la musica, que fueron pensadas en terminos explicativos, y validar su poder explicativo a traves 
de construir modelos generativos con esas teorias cognitivas y luego analizar las piezas generadas}}
\newline

\comment{Objetivo refinado t'ecnicamente: dada una pieza musical, componer una melod'ia que respete los criterios est'eticos definidos por esta. }

Tengo que acordarme de enmarcar todo esto dentro de la musica tonal


Si bien el trabajo de Lerdahl y Jackendoff describe un modelo potencialmente computacional, este no fue pensado para computar, sino para explicar los 
fen'omenos que ocurren en la psique de un sujeto expuesto a un est'imulo musical, es por esto que en el presente trabajo se tomar'an los aportes de estos autores
pero no se utilizar'a literalmente una gram'atica que explique el proceso de reducci'on que sufre un tema.
