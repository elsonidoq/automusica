\section{Rese~na hist\'orica y Motivaci\'on}
Hacia fines del siglo XIX y principios del siglo XX, con el prop'osito de estudiar la estructura subyacente de una pieza musical, 
Heinrich Schenker elabor'o una teor'ia de la coherencia tonal en la m'usica cuya derivaci'on en la praxis anal'itico-musical es conocida 
como \texttt{an'alisis schenkeriano}.
En su teor'ia, Schenker \footnote{Para profundizar en el tema refi'erase a \cite{forte03}} postula que la superficie musical 
(el conjunto de todos los sonidos que consituyen la pieza musical) es resultado de sucesivas transformaciones que sufre una estructura b'asica fundamental 
denominada por el autor la \emph{Ursatz}\footnote{Estructura fundamental en alem'an}. Estas transformaci'ones est'an descriptas en t'erminos de reglas de 
reescritura (dentro de la teor'ia llamadas t'ecnicas prolongacionales) que permiten, a partir de una cierta Ursatz, llegar a una pieza musical completa. 
Dado que la Ursatz es la estructura fundamental, 'esta s'olo contiene las notas que sostienen tonalmente a la pieza, que luego son elaboradas usando 
las reglas de reescritura.

Si bien la teor'ia schenkeriana fue concebido para llevar la Ursatz a una pieza musical completa, queda claro que se la puede utilizar tambi'en para estimar
cu'al podr'ia ser la Ursatz de un cierta superficie musical. De esta forma, al utilizar la teor'ia ``en sentido contrario'', en vez de hablar en t'erminos 
de \emph{prolongaciones} de la Ursatz a la superficie musical, habr'ia que hacerlo en t'erminos de \emph{reducciones}. Definida como lo opuesto a 
una prolongaci'on, una reducci'on permite especificar que una cierta nota dentro de un grupo es la m'as importante en t'erminos estructurales, 
y que el resto del grupo es una prolongaci'on.  Dado que esta relaci'on de reducci'on/prolonga'on se puede aplicar recursivamente, naturalmente emerge 
una relaci'on jer'arquica entre las notas de una pieza musical.  En esta jerarqu'ia, el nivel m'as bajo es lo que se encuentra escrito en la 
partitura, y a medida que se sube de nivel se encuentran diferentes versiones de la misma pieza musical cada vez menos prolongadas. En la figura 
\ref{fig_analisis_schenkeriano} se exhibe un ejemplo de reducci'on.


\begin{imagen}
    \file{images/schenkerian_example.png}
    \labelname{fig_analisis_schenkeriano}
    \desc{Ejemplo de un an'alisis utilizando la teor'ia de Schenker (tomado de \cite{Martinez01}).}
 %   \width{10cm}
\end{imagen}



%
%El parecido de este tipo de relaciones con las relaci'ones utilizadas por Noam Chomsky para analizar el lenguaje natural es notable. 
%Si bien en la teor'ia de Chomsky las relaciones son relaciones del tipo ``es un'', y en la teor'ia de Schenker son del tipo ``es una elaboraci'on de'', 
%estas se enmarcan matem'aticamente en el mismo lugar.  

En 1983, el m'usico Fred Lerdahl y el ling\"uista Ray Jackendoff publicaron 
\texttt{A Generative Theory of Tonal Music} \citep{LerdahlJackendoff83} en la que proponen una \emph{gram'atica} para analizar la m'usica en 
t'erminos parecidos a los de Schenker tomando una postura an'aloga a la tomada por la escuela generativa de la ling\"u'istica 
(mayormente conocida por el trabajo de Noam Chomsky, 1957, 1965). 

La escuela generativa de la ling\"u'istica trata de caracterizar el conocimiento que tiene el ser humano para ser capaz de producir y entender oraciones 
que nunca ha escuchado o emitido model'andolo con gram'aticas. Lerdhal y Jackendoff adaptan el formalismo para que sea adecuado para modelar 
la m'usica tonal, modelizando as'i los procesos perceptuales de un oyente musical idealizado. 
Lo interesante de este trabajo es el fundamento que se le d'a a la elecci'on de las reglas de la gram'atica, puesto que proponen una teor'ia 
de naturaleza computacional basada en los conceptos claves de la teor'ia schenkeriana y en una serie de supuestos cognitivos. 

Es importante diferenciar el significado con el que los autores utilizan el t'ermino ``generativo''\citep[p. 6]{LerdahlJackendoff83}. El significado que asignan 
a esta palabra tiene una ra'iz matem'atica queriendo decir ``Un modelo que describe un conjunto, posiblemente infinito, a partir
de una cantidad finita de reglas''. Por el contrario, en este trabajo, cuando se utilice el t'ermino ``generativo'' se lo utilizar'a significando 
que el modelo propuesto no s'olo describe este conjunto, sino que adem'as brinda un algoritmo para generar un elemento del mismo.

Si bien se podr'ia intentar transformar el modelo de Lerdahl y Jackendoff en un modelo generativo, existen otras teor'ias complementarias que describen
ciertos aspectos de la cognici'on musical. Es por esto que el enfoque de este trabajo ser'a tomar una serie de atributos aplicados a melod'ias 
bas'andose en estas teor'ias, 
y construir para cada una de ellas un modelo generativo teniendo particular cuidado
tanto en la aplicaci'on de los criterios que proponen las teor'ias como tambi'en en lograr una implementaci'on modular de cada modelo. 
Una vez construidos todos los modelos, ser'a posible analizar la m'usica generada por el conjunto de modelos al afectar individualmente uno en particular. Esto
permitir'a indirectamente analizar la validez y el impacto que tiene lo que propone la teor'ia cognitiva que dio origen al modelo.

Existen distintas aproximaciones para construir los modelos.  En este trabajo se utilizar'an t'ecnicas estad'isticas, es por esto que se contar'a 
con una fase de entrenamiento en donde se le presentar'a al sistema un archivo en formato MIDI, y este construir'a los modelos basado en la pieza 
de entrenamiento. Se opt'o por no entrenar los modelos con m'as de una obra musical para mantener el trabajo acotado: Combinar la informaci'on obtenida
de varios archivos MIDI presenta un problema en s'i mismo.

Para poder avanzar en pos del objetivo reci'en planteado, es necesario contextualizarlo, y luego sentar un vocabulario, para finalmente exhibir los
modelos propuestos. De esta forma, la secci'on \ref{sec:state_art} contextualizar'a este trabajo haciendo menci'on al estado del arte 
en el tema. En el cap'itulo \ref{cap:background} se introducir'a vocabulario musical y matem'atico 
para abordar en los cap'itulos \ref{cap:notenote} y \ref{cap:jerar} los modelos propuestos. Por 'ultimo, en el cap'itulo 5 se describe la metodolog'ia
experimental elegida, junto con una serie de experimentos y sus resultados.

