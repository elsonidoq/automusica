\section{Notas, alturas, acordes y tonalidades}
\label{sec_cogn_bg}
Esta secci'on cumple dos objetivos dentro de este trabajo; por un lado, sirve para el lector no familiarizado con teor'ia musical para establecer
ciertos conceptos generales que son de gran importancia en el desarrollo de este trabajo. Por otro lado, permitir'a ademas acentar un glosario
que se utilizar'a a lo largo de las diferentes secciones.
Principalmente, la m'usica se divide en dos grandes componentes u ejes: la del \texttt{tiempo} y la de la \texttt{altura}. 

La dimensi'on temporal responde a \emph{cu'ando} ocurre un evento, mientras que la altura expresa caracter'isticas sobre \emph{qu'e}
tipo de evento es.  Dejando fuera del an'alisis a los instrumentos de percusi'on, la altura es refiere a la percepci'on de un cierto sonido. 
Esta percepci'on si bien esta relacionada con caracter'isticas intr'insecas del sonido, como su intensidad, timbre y frecuencia, 
tambi'en hace referencia
al proceso cognitivo que ocurre por parte del oyente. De esta forma, si bien un sonido de una frecuencia de $440Hz$ es percibido como una nota 
$La$, tambi'en lo es un sonido de frecuencia $441Hz$.

Dentro de la m'usica occidental, existen criterios dentro de ambas componentes que ayudan a la organizaci'on de una pieza musical. 
En lo subsiguiente se dar'a cierto conocimiento general sobre estos criterios, introduciendo ciertos conceptos de la teor'ia musical, puesto que son necesarios
para abordar los fundamentos cognitivos y los modelos propuestos. 

Por 'ultimo, dado que a lo largo del presente trabajo se exhiben ejemplos, es necesario que el lector sepa comprender una partitura sencilla, de forma que se se 
explicar'a brevemente como ciertos conceptos introducidos se notan en el lenguaje musical.

\subsection{Sobre la organizaci\'on temporal}
El primer concepto importante a tener en cuenta, es que el tiempo es \emph{discreto}. Esto no quiere decir que a la hora de la producci'on musical lo sea, sino que 
en la abstracci'on musical, existe una unidad de referencia temporal, a la que se llama \emph{beat}. En torno a ella se organiza la estructura temporal, tocando
siempre notas que duren una fracci'on del mismo. De esta forma, para poder reproducir una partitura es necesario saber la duraci'on de un \emph{beat} (denominado
\emph{tempo}), luego el resto de las duraciones son una fracci'on de 'esta.

Siendo que una nota no puede tener una duraci'on arbitraria, y esta relacionada con la duraci'on del beat, existen s'imbolos y nombres para diferentes duraci'ones. 
En la figura \ref{fig:durations} se muestran las figuras correspondientes a notas y silencios de distintas duraciones. Se denomina plica, a la barra vertical que surje 
de la cabeza de las notas, con excepcion de la redonda. Es importante aclarar que cuando dos notas de duracion inferior o igual a una corchea se tocan juntas, 
se unen las plicas, m'as adelante se dar'a un ejemplo.

\begin{imagen}
    \file{images/figures.png}
    \labelname{fig:durations}
    \desc{Figuras de ejemplo}
    \width{12cm}
\end{imagen}

Las partituras luego se organizan en una serie de unidades denominadas \emph{comp'ases}. Todos los compases tienen la misma duraci'on, y esta se especifica
en la partitura por una fracci'on al comienzo de la misma. 'Esta fracci'on no debe interpretar de forma usual, ya que posee una sem'antica sutilmente diferente.
En la especificaci'on del comp'as, el denominador determina en que unidades se medir'a la longitud del comp'as, y el numerador determina cuantas unidades dura'ra.
De esta forma, un comp'as de $\frac{3}{4}$ indica que la unidad de referencia es la negra, y que hay $3$ negras por comp'as.
La figura \ref{fig:time_signatures} muestra los compases m'as comunes. 

\begin{imagen}
    \file{images/Common_time_signatures.png}
    \labelname{fig:time_signatures}
    \desc{Compases comunes}
    \width{5cm}
\end{imagen}

\alert{revisar}
Es importante notar a fines de notar la duraci'on del comp'as, no existe diferencia entre el comp'as de $\frac{4}{4}$ y el comp'as de $\frac{2}{2}$, su diferencia
radica en la acentuaci'on recibida por distintos eventos: En el comp'as de $\frac{4}{4}$, hay 4 eventos de negra, donde el primero se percibe como acentuado, 
el segundo y el cuarto como d'ebiles, y el tercero como semi acentuado, mientras que en el comp'as de $\frac{2}{2}$ hay solo dos eventos, uno acentuado y uno no 
acentuado. 

Adem'as de los s'imbolos explicados, existen ciertos s'imbolos que permiten que el lenguaje sea mas rico, y describir nuevas duraciones. El s'imbolo de inter'es
en este caso es el denominado \emph{puntillo}. El puntillo se nota como un peque~no punto que se coloca a continuaci'on de la cabeza de una nota, 
y determina que la duraci'on de esa nota debe multiplicarse por $\frac{3}{2}$. De esta forma, una \emph{negra con puntillo}, dura una negra y media. 

De esta forma, para fijar conceptos, la partitura de la figura \ref{fig:example_measure} las duraci'on del primer s'imbolo es $\frac{3}{4}=\frac{1}{2}\times\frac{3}{2}$,
la duraci'on del segundo s'imbolo es $\frac{1}{4}$, puesto que tiene dos plicas, y eso indica que es una semicorchea. Luego sigue un silencio de negra, y luego se 
repite. La letra $c$ al principio del comp'as es una abreviaci'on para decir $\frac{4}{4}$.

\begin{imagen}
    \file{images/simple_measure.png}
    \labelname{fig:example_measure}
    \desc{Ejemplo para fijar conceptos}
\end{imagen}

\subsection{Sobre la organizaci\'on de las alturas}
Dentro de toda pieza musical, existe una organizaci'on jer'arquica de las alturas. A nivel general, la primer entidad que permite organizar, y por lo tanto dar 
coherencia, es la tonalidad. La tonalidad establece un marco de referencia interpretativo  

Cuando se refiere a las altur
De la misma forma que al comienzo de una partitura se debe especificar la duraci'on de los compases, y al mismo tiempo como se regular'an los acentos m'etricos,
tambi'en debe especificarse cuestiones relacionadas con 

Si bien a lo largo de este trabajo se utilizar'a solo la clave de sol en la tonalidad de do mayor, se explicar'a brevemente la utilidad de la clave y de la armadura
de clave.
El primer elemento a explicar, es la llamada clave. Si bien 

definir modulacion, cambio de tonalidad 
