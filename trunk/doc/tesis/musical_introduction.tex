\section{Sobre notas, alturas, acordes y tonalidades}
\label{sec:musical_intro}
Esta secci'on tiene dos objetivos; por un lado, que el lector no familia\-rizado con teor'ia musical conozca
ciertos conceptos generales que son utilizados en el desarrollo de este trabajo y, por el otro, establecer un glosario
que se utilizar'a a lo largo de las diferentes secciones.

Principalmente, la m'usica tonal opera sistem'aticamente sobre dos grandes dimensiones: el \texttt{tiempo} y la \texttt{altura}. 
La dimensi'on temporal responde a \emph{cu'ando} ocurre un evento, mientras que la altura expresa caracter'isticas sobre \emph{qu'e}
tipo de evento es. Por altura se refiere a la localizaci'on de un sonido en una escala tonal, dependiendo de la velocidad de las vibraciones de la fuente sonora; 
as'i las frecuencias r'apidas producen sonidos m'as agudos, y las lentas sonidos m'as graves \citep[p. 565]{kennedy1996oxford}.
Sin embargo, aunque este atributo est'a relacionado con caracter'isticas intr'insecas del sonido, como su intensidad, timbre y frecuencia, 
tambi'en hace referencia
al proceso cognitivo que ocurre por parte del oyente. De esta forma, si bien sonidos de $440\mbox{Hz}$ y $441\mbox{Hz}$ son distintos, 
perceptualmente se perciben de la misma forma, como la nota La.

En la m'usica occidental, existen criterios dentro de ambas dimensiones que ayudan a la organizaci'on de una pieza musical. 

%Por 'ultimo, dado que a lo largo del presente trabajo se exhiben ejemplos, es necesario que el lector sepa comprender una partitura sencilla, 
%de forma que se se explicar'a brevemente c'omo ciertos conceptos introducidos se anotan con el c'odigo convencional de escritura musical.

\subsection{Sobre la organizaci\'on temporal}
\label{sec:temporal_organization}

El primer concepto importante a tener en cuenta es que en la m'usica tonal el transcurrir del tiempo est'a \emph{discretizado} en unidades de referencia 
temporal, cada una denominada \emph{beat}. En torno a ella se organiza la estructura temporal, tocando
siempre notas que duren una fracci'on del mismo. De esta forma, para poder reproducir una partitura es necesario conocer 
la duraci'on del \emph{beat} de referencia, denominada \emph{tempo}, luego el resto de las duraciones son una fracci'on de 'esta.


\citet*{LerdahlJackendoff83} hacen una distinci'on entre dos estructuras temporales que ocurren en simultaneidad en la m'usica tonal:
la estructura \emph{m'etrica} y la estructura del \emph{agrupamiento}\footnote{\emph{meter} y \emph{grouping} en ingl'es}. 
La estructura del agrupamiento hace referencia a la organizaci'on de una pieza musical en unidades discursivas que pueden ser motivos, frases, 
secciones, etc. 
Cada una de estas unidades, es denominada por los autores \emph{grupo}. Asimismo, el oyente infiere una estructura \emph{regular} de beats. 
Algunos beats reciben una acentuaci'on mayor que otros, determinando lo que los autores definen como la estructura m'etrica. Dentro de esta estructura
existen distintos niveles, cada uno determinado por el tiempo transcurrido entre dos beats consecutivos. 
As'i, el contexto m'etrico de una pieza refiere a la estructura m'etrica inferida a partir de esta.

Ambas estructuras son jer'arquicas, en el sentido de que existen sub-estructuras a distintos niveles, y que las sub-estructuras de niveles superiores 
incluyen por completo a las de nivel inferior
\footnote{Para una definici'on m'as formal de la jerarqu'ia referirse a \citet[cap. ~2]{LerdahlJackendoff83}}. De esta forma, una secci'on
de una pieza musical estar'a formada por una sucesi'on de frases. Cada frase pertenecer'a s'olo a una secci'on aunque distintas repeticiones de la misma
frase pertenecer'an potencialmente a distintas secciones. Obs'ervese que con esta definici'on, los niveles altos en la jerarqu'ia abarcan 
porciones m'as largas en la pieza. 

Respecto a la estructura m'etrica, la noci'on de jerarqu'ia es menos evidente. En la definici'on original, los autores utilizan una representaci'on geom'etrica
por el simple hecho que una \emph{m'etrica} tiene sentido cuando hay algo que medir. De esta forma, utilizando esta met'afora geom'etrica, los autores representan 
a los beats como puntos sin duraci'on. Luego definen el concepto de \emph{time-span} como el tiempo entre dos beats sucesivos. 
La relaci'on entre un beat fuerte, y su posici'on en la jerarqu'ia m'etrica es simplemente la siguiente: si un beat se escucha m'as fuerte que el siguiente, eso implica
que el primero est'a en un nivel mayor que el segundo.

%\red{este parrafo no est'a bien, lo dejo para acordarme de reescribirlo}
%
%La organizaci'on jer'arquica de la m'usica no es una teor'ia solo postulada por Lerdahl y Jackendoff; Cooper y Meyer (\cite{CooperMeyer60}) plantean una teor'ia similar en
%estos t'erminos, Kramer (\cite{Kramer88}) si bien no habla directamente de una jerarqu'ia, propone que tienen un rol 
%estructurante\alert{tengo que leer m'as, as'i se mejor de lo que hablan estos dos muchachos}. 
%
%Un nivel de especial inter'es, es el denominado \emph{tactus}, que b'asicamente es el marcado por el director de orquesta al mover su batuta. 
%El tactus tambi'en es la distancia entre los beats que el oyente marca cuando mueve el pie y est'a relacionado con el baile. 
%
Es importante notar que la estructura temporal es ambigua, en el sentido de que muchas veces no existe un 'unico an'alisis de una pieza musical 
en una estructura m'etrica y de agrupamiento.

\begin{imagen}
    \file{images/metrical_structure2.png}
    \labelname{fig:metrical_structure}
    \desc{An'alisis de estructura m'etrica y de agrupamiento. Imagen tomada de \cite{LerdahlJackendoff83}. Los par'entesis inferiores representan
    a la estructura del agrupamiento. Los puntos representan la jerarqu'ia m'etrica: columnas con mayor cantidad de puntos representan un acento a un nivel 
    mayor.}
    \width{10cm}
\end{imagen}

En la figura \ref{fig:metrical_structure} se exhibe una partitura con un an'alisis m'etrico y de agrupamiento. 
No se pretende que el lector interprete esta figura por completo, sino mostrar c'omo estos dos an'alisis se aplican a una pieza musical. 
En esta figura se puede observar c'omo cada nivel de la estructura m'etrica determina un pulso regular, 
y los niveles inferiores subdividen a este tambi'en de forma regular. Tambi'en se puede ver la jerarqu'ia determinada por los dos an'alisis.


En cuanto a la duraci'on de las notas, dado que cada nota no puede tener una duraci'on arbitraria, y que esta duraci'on est'a relacionada con el tempo
a trav'es de una fracci'on, existen s'imbolos y nombres para diferentes duraci'ones. 
En la figura \ref{fig:durations} se muestran las figuras correspondientes a notas y silencios de distintas duraciones. De las figuras exhibidas, la relaci'on de duraci'ones
es siempre del doble con respecto a otra: la redonda dura el doble que una blanca, la blanca dura el doble que una negra, etc. 
Se denomina plica, a la barra vertical que surge 
de la cabeza de las notas, con excepci'on de la redonda. Es importante aclarar que cuando dos notas de duraci'on inferior o igual a una corchea se tocan seguidas, 
se unen las plicas (ver figura \ref{fig:example_measure}).

\begin{imagen}
    \file{images/note_values.jpg}
    \labelname{fig:durations}
    \desc{Figuras de ejemplo}
    \width{10cm}
\end{imagen}

Las partituras luego se organizan en una serie de unidades, cada una denominada \emph{comp'as}. Todos los compases tienen la misma duraci'on, y 'esta se especifica
en la partitura por una fracci'on al comienzo de la misma. Esta fracci'on no debe interpretar de forma usual, ya que posee una sem'antica sutilmente diferente.
En la especificaci'on del comp'as, el denominador determina en qu'e unidades se medir'a la longitud del comp'as, y el numerador determina cu'antas unidades 
durar'a. El valor de referencia para el denominador es el $4$ refiriendo a una negra, y m'ultiplos de este refieren a m'ultiplos de una negra. Por ejemplo
$8$ refiere a corcheas.  De esta forma, un comp'as de $\frac{3}{4}$ indica que la unidad de referencia es la negra, y que hay $3$ negras por comp'as.
La figura \ref{fig:time_signatures} muestra c'omo se especifican en una partitura los compases m'as comunes.

\begin{imagen}
    \file{images/Common_time_signatures.png}
    \labelname{fig:time_signatures}
    \desc{Compases comunes}
    \width{5cm}
\end{imagen}

Obs'ervese que a fines de notar la duraci'on del comp'as, no existe diferencia entre el comp'as de $\frac{3}{4}$ y el comp'as de $\frac{6}{8}$, su diferencia
radica en la acentuaci'on recibida por distintos eventos: En el comp'as de $\frac{3}{4}$, existen 3 eventos de negra, donde el primero se percibe como acentuado, 
y los dos siguientes se perciben como m'as d'ebiles, mientras que en el comp'as de $\frac{6}{8}$ hay 6 eventos, donde el primero es el m'as 
acentuado, luego sigue el cuarto, y por 'ultimo el resto de los beats. En la figura \ref{fig:acentos_compas} se exhiben los compases de $\frac{3}{4}$, $\frac{4}{4}$ 
y $\frac{6}{8}$ con sus respectivos an'alisis m'etricos. En esta figura, los niveles corresponden a distintos \emph{time-spans}, es decir, los beats 
que ocurren en el tercer nivel tienen un per'iodo mayor que los beats que ocurren en los niveles uno y dos. En el comp'as de $\frac{4}{4}$, 
el nivel inferior corresponde a una negra, sin embargo, en los compases de $\frac{3}{4}$ y de $\frac{6}{8}$ el nivel inferior corresponde
a una corchea. N'otese que en el caso del comp'as de $\frac{3}{4}$, la unidad de referencia es la negra (por ser el denomidador el n'umero 4), sin embargo
en la figura se muestra el an'alisis m'etrico partiendo del nivel de corchea para poder compararlo con el an'alisis m'etrico del comp'as de $\frac{6}{8}$.

\begin{imagen}
    \file{images/metrical_accents}
    \labelname{fig:acentos_compas}
    \desc{Acentuaci'on m'etrica inferida para los compases $\frac{3}{4}$, $\frac{4}{4}$ y $\frac{6}{8}$. Im'agen adaptada de \cite{Temperley2007}.}
    \width{10cm}
\end{imagen}

Adem'as de los s'imbolos explicados, existen ciertos s'imbolos que permiten que el lenguaje sea m'as rico, y describir nuevas duraciones. Uno de ellos
es la ligadura, que permite sumar la duraci'on de dos figuras. Esta se nota como un arco que une dos notas que tienen la misma altura. 
%Un caso particular de la ligadura, es el puntillo. El puntillo se nota como un peque~no punto que se coloca a 
%continuaci'on de la cabeza de una nota, y determina que la duraci'on de esa nota debe multiplicarse por $\frac{3}{2}$. 
%De esta forma, una \emph{negra con puntillo}, dura una negra y media. 

Para fijar conceptos, se exhibe una partitura de ejemplo en la figura \ref{fig:example_measure}. 
La duraci'on del primer y segundo s'imbolo es $\frac{1}{2}$,
luego sigue un silencio de negra, cuyo onset (momento donde empieza) es $1$. Seguido al silencio de negra, una negra ligada a una semi corchea, 
es decir, su duraci'on ser'a $1 + \frac{1}{4}$, 
luego sigue una semi corchea y un silencio de corchea, completando
con la duraci'on total de 4 negras. La letra $c$ al principio del comp'as es una abreviaci'on para decir $\frac{4}{4}$.

\begin{imagen}
    \file{images/pedagogic_measure.png}
    \labelname{fig:example_measure}
    \desc{Ejemplo para fijar conceptos}
    \width{12.5cm}
\end{imagen}

\subsection{Sobre la organizaci\'on de las alturas}
La m'usica tonal, es llamada as'i debido a que un tono (o altura) es tomado como punto de referencia.
Este punto de referencia es una nota llamada \emph{t'onica} y es la que le da nombre a la escala; por ejemplo Do es la t'onica de la escala de Do mayor. 

Existen dos formas alternativas de denominar a las notas, la usual que utiliza los nombres Do, Re, Mi, Fa, Sol, La y Si y la americana, que asigna
letras a cada nota resultando en C, D, F, E, G, A y B respectivamente. En general en los gr\'aficos se utilizar'a la notaci'on americana 
por ser esta m'as corta.

Las escalas mayores y menores de la m'usica occidental, llamadas escalas diat'onicas, son un subconjunto de siete alturas de la escala crom'atica.
La escala crom'atica es aquella que divide la octava (la porci'on de espectro que existe entre una frecuencia $x$ y una $2x$)
en doce partes iguales.
En la tabla \ref{tab:cromatica} se exhibe las frecuencias de las notas de la escala crom'atica comenzando desde $La$. 

Obs'ervese que s'olo existen 7 s'imbolos, y existen 12 alturas a ser nombradas. Para construir los restantes se yuxtaponen otros dos s'imbolos, 
seg'un sea el caso. 
El \emph{bemol} ($\flat$) permite especificar que una nota es un semitono m'as grave. El sostenido ($\sharp$) hace lo mismo en sentido contrario, 
es decir, aumenta un semitono.
Esto hace que no sea 'unica la forma de referirse a una nota, siendo $La\sharp = Si\flat$, o $Mi\sharp=Fa\flat$. Existe un criterio para determinar la 
unicidad, pero antes es necesario definir m'as precisamente el concepto de tonalidad o escala.


\begin{figure}
\begin{center}
    \begin{tabular}[c]{|l|c|}
    \hline
    \textbf{Nota} & \textbf{Frecuencia en $\mbox{Hz}$} \\
    \hline 
    La		        &	440.00 \\
    La$\sharp$		&	466.16 \\
    Si	        	&	493.88 \\
    Do	        	&	523.25 \\
    Do$\sharp$      &	554.37 \\
    Re	        	&	587.33 \\
    Re$\sharp$      &	622.25 \\
    Mi	        	&	659.26 \\
    Fa	        	&	698.46 \\
    Fa$\sharp$      &	739.99 \\
    Sol	        	&	783.99 \\
    Sol$\sharp$     &	830.61 \\
    La	        	&	880.00 \\ 
    \hline
    \end{tabular}
 \caption{Frecuencias en Hertz de una octava comenzando de La $440\mbox{Hz}$.}
 \label{tab:cromatica}
\end{center}
\end{figure}


Utilizando la escala crom'atica se puede definir la noci'on de \emph{semitono} y de \emph{intervalo mel'odico}. Un semitono es la distancia
entre dos notas consecutivas dentro de la escala crom'atica. Por ejemplo Mi y Fa est'an a un semitono de distancia. Un intervalo mel'odico, 
es la distancia entre dos notas medida en semitonos. En la teor'ia musical, a estos intervalos se les pone un nombre que permite recordarlos 
facilmente, sin embargo a lo largo de este trabajo no se utilizar'an estos nombres y se los nombrar'a como la cantidad de semitonos que separa
a las notas que lo conforman. Por ejemplo, las notas La y Si est'an a 2 semitonos de distancia, por lo que el intervalo mel'odico
se notar'a como \IM{2}. 

Como se mencion'o al principio de esta secci'on, de la escala crom'atica se elije un subconjunto de 7 notas. Estas notas ser'an referenciadas
respecto de una nota normativa, la t'onica. El tipo de escala con la que se est'e trabajando depender'a del patr'on de intervalos que 
forman las 7 notas elegidas respecto esta nota normativa. Por ejemplo, la escala mayor forma el siguiente patr'on de interv'alos respecto a la t'onica:

\begin{center}
\IM{2} \IM{4} \IM{5} \IM{7} \IM{9} \IM{11} \IM{12}
\end{center}

Si se observa el intervalo que forma cada nota respecto de la anterior, el patr'on resultante es el siguiente

\begin{center}
\IM{2} \IM{2} \IM{1} \IM{2} \IM{2} \IM{2} \IM{1}
\end{center}

Si se rota este patr'on dos lugares hacia la derecha, se obtiene el patr'on que forman las escalas menores:

\begin{center}
\IM{2} \IM{1} \IM{2} \IM{2} \IM{1} \IM{2} \IM{2} 
\end{center}

Definida la tonalidad, el criterio de unicidad para nombrar las notas es que debe aparecer una y s'olo una vez el nombre de cada nota. Siendo as'i,
la escala mayor de Do quedar'a formada por las notas Do, Re, Mi, Fa, Sol, La, Si, mientras que la escala menor de Do estar'a formada
por las notas Do, Re, Mi$\flat$, Fa, Sol, La$\flat$, Si$\flat$. Obs'ervese que ser'ia incorrecto escribir Re$\sharp$ en lugar de Mi$\flat$ en la
escala de Do menor puesto que el Re ya ha sido nombrado.

%\red{Greg, decime si se entiende esto, si no se entiende trata de decirme que cosa no entendes as'i despues la trabajo con favio\ldots la onda es que como no se tanto de esto vamos a armar estos parrafos juntos en base a lo que entienda alguien que no sabe de musica}
Por 'ultimo, la entidad restante a hacer menci'on es el \emph{acorde}. Un acorde est'a formado por tres o m'as notas. Cada acorde abarca un determinado
lapso temporal en la obra musical, y dentro de este lapso ser'a el foco arm'onico. Los acordes pueden ser de dos tipos: diat'onicos o crom'aticos. 
Los diat'onicos se forman con las notas de la escala por superposici'on de dos intervalos de 3 o 4 semitonos, seg'un correspondan, a partir de la nota 
considerada como base o ``fundamental'' y que le da su nombre al acorde. 
Esto no quiere decir que que siempre los acordes se conformen por saltos de 3 o 4 semitonos, sino que estos saltos son tomados como punto de referencia
de la misma forma que la t'onica es tomada como punto de referencia en la tonalidad.
De esta forma, dentro de la escala de Do mayor el acorde diat'onico formado a partir de Sol ser'a Sol mayor, cuyas notas son Sol, Si y Re, cuyos intervalos
son \IM{4}, \IM{3} (ver tabla \ref{tab:cromatica}). 

La armon'ia postula que la m'usica es una sucesi'on de tensiones y resoluciones. En este marco, la funci'on que cumple un acorde, es generar una condiciones
detensi'on 'o resoluci'on dentro de la tonalidad. Dado que lo importante en una escala es el patr'on de intervalos que se forman entre sus notas, y 
no cu'al es la t'onica\footnote{Es decir, para el an'alisis, es lo mismo estar trabajando con la escala de Re menor o la escala de Do mayor}, en general en el estudio de la armon'ia se abstrae esto y se nombra a los acordes en relaci'on al intervalo que forma su
fundamental con la t'onica. De esta forma, el acorde de Sol mayor en la tonalidad de Do pasa a llamarse acorde de quinta, puesto que la nota Sol
es la quinta nota de la escala de Do. Cuando se utiliza notaci'on escrita, se utilizan n'umeros romanos, por lo tanto el acorde reci'en mencionado
se llamar'a V. 

%La tensi'on que generan los acordes est'a altamente relacionada con los intervalos que estos llevan dentro (los grupos de notas tomados de a pares).
%Todo gira en torno al intervalo del tritono, que como su nombre indica tiene 3 tonos, es decir 6 semitonos de longitud. Este intervalo genera una 
%tensi'on que se resuelve en una tercera mayor(\IM{4}). Dentro de una escala mayor o menor, hay un 'unico tritono. Por ejemplo, en la escala de Do mayor, 
%las 'unicas dos notas de esta escala que forman este intervalo son B y F y estas \emph{resuelven} en el intervalo de tercera mayor formada por
%C y E. De esta forma, un acorde que tenga las notas B y F generar'a una fuerte tensi'on hacia un acorde que tenga las notas C y E. Si un acorde
%tiene alguna de las dos notas del tritono generar'a una tensi'on parcial. De esta forma, los acordes se clasifican en tres tipos: de t'onica,
%de subdominante y de dominante. Los acordes de dominante, son aquellos que tienen el tritono dentro de sus notas, los de subdominante, son los 
%que tienen al menos una nota del tritono, y los de t'onica, no solo no tienen ninguna nota del tritono, sino que deben tener la t'onica
%de la escala.


Tornando el foco a la lectura musical, existen dos elementos que permiten dotar de significado las l'ineas y espacios del pentagrama: la clave
y la armadura de clave. La clave asigna una nota a un rengl'on en particular, y la armadura permite especificar en que escala se tocar'a. 
Dado que a lo largo de este trabajo solo se dar'an ejemplos sobre la tonalidad de Do mayor, se obviar'a la explicaci'on de que es la armadura de clave. 
La clave utilizada para todos los ejemplos de esta tesis ser'a la clave de sol que determina que la segunda l'inea, contando de abajo hacia
arriba, ser'a la nota Sol. El espacio entre esta l'inea y la siguiente, ser'a la nota La, y as'i sucesivamente. Si se deseara especificar
la nota La$\sharp$, basta con anteponer el s'imbolo $\sharp$ a la cabeza de la nota. En la figura \ref{fig:escala_mayor} se exibe la escala de Do mayor.


\begin{imagen}
    \file{images/major_scale.png}
    \labelname{fig:escala_mayor}
    \desc{Escala mayor de Do}
    \width{12cm}
\end{imagen}
