
\newcommand{\po}{\ensuremath{po}}
\newcommand{\PitchOnsets}{\ensuremath{\po_1 \ldots \po_n}}

\newcommand{\ro}{\ensuremath{ro}}
\newcommand{\RestOnsets}{\ensuremath{\ro_1 \ldots \ro_k}}

\newcommand{\go}{\ensuremath{go}}
\newcommand{\GeneralOnsets}{\ensuremath{\go_1 \ldots \go_{n+k}}}

\newcommand{\Pitches}{\ensuremath{p_1 \ldots p_n}}

\newcommand{\Intervals}{\ensuremath{i_1 \ldots i_{n-1}}}

\newcommand{\Score}{\ensuremath{<PO, RO, P, m>}}

\section{Definiciones}
\subsection{Partituras}
Se define como partitura $P = \Score$ a una 4-upla:
\begin{itemize}
    \item Un n\'umero $m$, que es el numerador del compas
    \item Una secuencia $PO = \PitchOnsets$ de onsets en donde suena un pitch 
    \item Una secuencia $RO = \RestOnsets$ de onsets en donde hay un silencio 
    \item Una secuencia $P = \Pitches$ de pitches asociados a $PO$ 
\end{itemize} 

Una partitura $S = \Score$ bien formada debe cumplir
\begin{itemize}
    \item $\forall i, i<n, \po_i \leq \po_{i+1}$\\
        intuici\'on: la lista esta ordenada, pero varias notas pueden sonar en simult\'aneo
    \item $\forall j, j<k, \ro_j < \ro_{j+1}$
    \item $\forall i,j, i<n \land j<k, \po_i \neq ro_j$ 
\end{itemize}

Por notaci\'on se llama \Intervals a la secuencia de intervalos, es decir $i_i = <p_i, p_{i+1}>$.\newline

Se define adem\'as la lista de onsets en general como $GO = \GeneralOnsets$, que es la secuencia
que resulta de unir $RO$ y $PO$ y luego ordenarlos.
