\documentclass[a4paper,10pt]{article}
\usepackage[spanish,activeacute]{babel}


%opening
\title{Markov para la ritmica}

\begin{document}
\maketitle
\section{El modelo}
El modelo est\'a basado en cadenas de markov. Al procesar un tema, lo que se hace es primero es estimar la metrical 
grid, y luego elegir un nivel de la metrical grid para dividir el tema.
Una vez elegido el nivel, dado que todos los beats del mismo nivel deben de estar equidistantes, se calcula el 
tiempo efectivo en el tema que hay entre dos beats consecutivos del nivel deseado, llamemoslo $t$.
De esta forma se generan potencialmente tantos estados como numeros naturales haya menores a $t$ (y se los etiqueta con este n\'umero).
Luego para colocar las aristas entre los nodos se recorre la lista de notas, y desde el nodo actual, llamemoslo
$n$, (que inicialmente es el nodo etiquetado como 0) se agrega una transicion al nodo $n+k$ modulo $t$,
donde $k$ es la duracion de la nota en cuestion.
Una vez terminado esto, puede ser que no sea posible hacer una random walk porque haya un estado (el final) que no
tiene transiciones salientes, de esta forma se agrega una transicion desde este estado hasta el estado mas cercano
que tenga transiciones salientes.

Para generar un compas sencillamente se recorre la cadena de markov, hasta que el estado actual sea menor que el proximo estado determinado por la cadena

\end{document}
