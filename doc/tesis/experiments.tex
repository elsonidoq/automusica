\section{Introducci\'on}
En este cap'itulo se presentan una serie de experimentos psicol'ogicos que se realizaron con el objeto de validar cognitivamente los modelos propuestos.

Para llevarlos a cabo hubo que 
realizar una implementaci'on donde los modelos puedan operar de forma aislada, garantizando que las 'unicas dependencias 
entre ellos sean las exhibidas en el figura \ref{fig:arquitectura}. Dejando de lado cuestiones implementativas, 
fue necesario tambi'en analizar y decidir el m'etodo a utilizar para llevar a cabo el experimento y luego evaluar los resultados.

Siendo as'i, este cap'itulo se estructurar'a de la siguiente forma: primero se detallar'an cuestiones relacionadas con 
la construcci'on de los experimentos, luego los experimentos en s'i, planteando en cada hip'otesis a poner a prueba y los
resultados obtenidos.
Por 'ultimo se realizar'a una discusi'on de los resultados respecto al trabajo en general.

\subsection{Agentes independientes}
Como se explic'o en la introducci'on, el concepto de de \emph{agente independiente} es fundamental para controlar las variables. 
Cada algoritmo recibe una entrada y produce una salida. Los algoritmos se afectan entre s'i a partir de las 
interacciones dadas por la entrada/salida que producen, sin embargo, la implementaci'on interna de cada uno
no debe afectar. 

En general, la forma generar un comportamiento que parezca aleatorio es mediante un \emph{generador de n'umeros pseudo-aleatorios}. 
Este generador es b'asicamente una lista infinita de n'umeros que estad'isticamente se comportan como si fueran uniformes. 
De esta forma el software puede utilizar esta secuencia, consult'andola cada vez que deba tomar un comportamiento no determin'istico. 
Cada consulta descarta un n'umero de esta lista, haciendo que la pr'oxima consulta lea el siguiente n'umero. 
Como las computadoras son determin'isticas, para que la secuencia de n'umeros cambie cada vez que se ejecute el programa se utiliza un n'umero 
llamado \emph{semilla} (o \emph{seed} en ingl'es) para inicializar la secuencia (que puede ser, por ejemplo, la hora actual). 
En el lenguaje en el que se desarroll'o la aplicaci'on (Python\footnote{http://www.python.org}), el generador por defecto de n'umeros aleatorios 
es global, es decir, todos los algoritmos estar'ian consultando al mismo generador. De esta forma, cambios en cualquier algoritmo que impliquen en 
una consulta m'as o una consulta menos indirectamente afectar'an al resto de los algoritmos.

Es por esto que un efecto que es necesario aislar es la cantidad de consultas que hace cada algoritmo al
generador de n'umeros aleatorios. Es por esto que cada agente cuenta con su propio generador de n'umeros aleatorios.

\subsection{Posibles experimentos}
En esta secci'on se presentan una serie de posibles experimentos a realizar con los modelos propuestos en este trabajo. 
No se abundar'a en una descripci'on exhaustiva de cada uno, s'olo se enunciar'a cu'al es la hip'otesis a probar.
Se llevar'an a cabo s'olo algunos para ilustrar la metodolog'ia de validaci'on propuesta.

\begin{itemize}
  \item Validaci'on del modelo de acento m'etrico:
    \begin{enumerate}
      \item Analizar si el modelo descripto en la secci'on \ref{sec:metric_model} es capaz de generar r'itmicas que hagan a un oyente inferir la misma 
            estructura m'etrica que la pieza original. 
      \item Analizar el impacto que tiene la utilizaci'on de distintos valores para el per'iodo en la inferencia por parte del escucha de la estructura m'etrica.
%      \item Analizar en qu'e punto los acentos fenom'enicos son necesarios para la inferencia del acento m'etrico: El modelo propuesto no contempla su 
%      existencia y los midis utilizados tampoco conllevan esa informaci'on.

    \end{enumerate}

  \item Validaci'on del modelo de inferencia de la tonalidad: 
    \begin{enumerate}
      \item Analizar si el modelo de los \emph{pitch profiles} es suficiente como para inducir en el oyente la tonalidad que se desea inducir. 
      \item Analizar si deber'ia distinguirse entre las notas del acorde, puesto que en la teor'ia de \cite{Lerdahl2001} son distinguidos pero no lo son en 
      la modelizaci'on propuesta en la secci'on \ref{sec:harmonic_context_model}.
    \end{enumerate}

  \item Validaci'on del modelo de fraseo:
    \begin{enumerate}
      \item Analizar si el modelo de las frases mejora al modelo sin las frases.
      \item Analizar la diferencia entre melod'ias generadas con distintos valores para los percentiles de los predicados $M$ y $S$ (ver secci'on \ref{sec:phrase_model}).
    \end{enumerate}


  \item Validaci'on del modelo de elaboraci'on mot'ivica.
    \begin{enumerate}
      \item Analizar si el modelo con elaboraciones mot'ivicas mejora al modelo sin elaboraciones mot'ivicas.
      \item Analizar las diferencias entre melod'ias generadas con distintos valores para $\alpha$ (ver secci'on \ref{sec:crp_model}).

    \end{enumerate}



\end{itemize}

Varios de los experimentos enunciados surgieron al analizar los resultados de los experimentos conducidos en las secci'ones \ref{sec:exp_percentiles}, 
\ref{sec:exp_frase}.

\subsection{Experimento con los valores del percentil asociado al predicado S}
\label{sec:exp_percentiles}
\subsubsection{Objetivo}
En la secci'on \ref{sec:must_predicates} se defini'o un algoritmo para organizar la melod'ia generada en t'ermino de unidades discursivas. Cada una de estas
unidades tienen la duraci'on de un acorde, y su meta es generar expectativas para el acorde siguiente (en la melod'ia, representado por una nota de anclaje
o apoyo). Se utiliza como herramienta para generar expectativas la distribuci'on en conjunto definida por el pitch profile afectado con el acorde que 
gobierna la sonoridad del momento y la distribuci'on de probabilidades que surge a partir de la teor'ia de la implicaci'on realizaci'on. Es posible 
utilizar esta distribuci'on como herramienta para generar expectativas puesto que tiene una interpretaci'on dual: adem'as de la propia de una distribuci'on 
de probabilidades, tiene la que corresponde a los fundamentos psicol'ogicos que se le dio. De esta forma notas muy probables son adem'as estables y 
fuertemente implicadas. Para capturar este hecho, se restringi'o la distribuci'on de probabilidades con un percentil, y se defini'o un criterio para que la 
frase sea v'alida: La nota de apoyo de la pr'oxima frase tiene que ser fuertemente implicada y estable.

Siendo asi, este experimento se propone testear la plausibilidad de las melod'ias generadas de acuerdo a diferentes percentiles (ver secci'on \ref{sec:must_predicates})
con el objeto de encontrar un valor de percentil adecuado para restringir la distribuci'on de probabilidad sobre la que el programa selecciona el 
intervalo implicativo a las notas de apoyo.

\subsubsection{Sujetos}
43 personas (20 mujeres y 23 varones) fueron convocadas voluntariamente por correo electr'onico para realizar la tarea. La media de edad fue de 34,23 a'nos 
(m'inima 19 y m'aximo 64 a'nos). Fueron agrupadas de acuerdo con tres niveles de formaci'on musical sistem'atica: No m'usicos (sin formaci'on musical; 16 sujetos), 
Formaci'on Moderada, (hasta 10 a'nos de formaci'on musical sistem'atica; 11 sujetos), M'usicos profesionales (m'as de 10 a'nos de formaci'on musical sistem'atica; 16 sujetos).

\subsubsection{Est\'imulos}
Se seleccionaron 5 fragmentos de piezas para piano del repertorio acad'emico cl'asico: (1) Danza Alemana WoO 13 N 11 de Beethoven; (2) Contradanza WoO 14 N 2 de Beethoven;
(3) Contradanza WoO 14 N 8 de Beethoven; (4) Danza Alemana D973 N 1 de Schubert; y (5) Danza Alemana D974 N 2 de Schubert. Estas piezas fueron utilizadas como input del 
programa. Cada uno de ellos gener'o 4 melod'ias de la extensi'on de la pieza original (aproximadamente 30 segundos) de acuerdo a 4 valores diferentes para el percentil 
que restringe la probabilidad de que las dos notas anteriores a cada punto de apoyo sean elegidas de acuerdo al modelo de Implicaci'on-realizaci'on mel'odica de 
E. Narmour. Los valores del percentil elegidos fueron 0; 0,3; 0,8 y 1. De tal modo en las melod'ias generadas con la indicaci'on del percentil 0 las notas de apoyo 
eran m'as probablemente abordada desde un par de notas anteriores siguiendo la teor'ia de Narmour. Por el contrario, las melod'ias generadas con el percentil 1 no 
presentaban para nada esa restricci'on. Los restantes componentes del modelo de generaci'on de melod'ias fueron mantenidos fijos para todas las melod'ias. Las melod'ias 
fueron ejecutadas por la computadora con valores de duraci'on y velocidad (sonoridad) normalizados de acuerdo a lo estipulado en la partitura (no expresivo), 
con un timbre de piano.

\subsubsection{Procedimiento}
Los sujetos escuchaban cada una de las 20 melod'ias y ten'ian que evaluarlas de acuerdo a sus propios criterios de ``buena forma'' en una escala de 5 puntos. 
Los sujetos entraban a la p'agina web a trav'es de un link que les era suministrado v'ia correo electr'onico. All'i se les comunicaba los fundamentos y objetivos 
de la prueba y se les proporcionaba ejemplos de la tarea de composici'on del programa. A continuaci'on se les explicaba el procedimiento y se le proporcionaban 4 
ejemplos de aprestamiento. Seguidamente se suced'ian los 20 'items de la prueba. Cada sujeto determinaba el momento de escuchar cada ejemplo clickeando sobre un 
'icono de ``play'', y colocaba su puntuaci'on directamente en un dispositivo ad hoc dise'nado en la p'agina. Al finalizar la prueba respond'ian un 
cuestionario de datos personales. 

\begin{imagen}
    \file{images/screenshot_experimento_estrellitas.png}
    \labelname{fig:rank_webi}
    \desc{Interfaz web utilizada para asignar calificaciones a melod'ias. Haciendo click en el 'icono de play es posible
    escuchar la melod'ia. Una vez finalizada la melod'ia, aparecen por debajo las estrellas donde es posible establecer un
    puntaje para la melod'ia reci'en escuchada.}
    \width{5cm}
\end{imagen}

\subsubsection{Resultados}
Se ejecut'o un modelo de an'alisis de la varianza (ANOVA) de mediciones repetidas \footnote{Utilizando el software SPSS} tomando la pieza input (5 niveles) 
y los percentiles de acuerdo a los que se generaron las melod'ias 
target (4 niveles) como factores intra-sujetos. Se consideraron los grupos de experiencia musical como factor entre-sujetos. 
Solamente el factor Pieza result'o significativo (F[4-39]=4.429; p=.002). Lo que m'as interesa aqu'i, el factor Percentil arroj'o una significaci'on marginal 
(F[3-40]=2.585; p=.056), sin embargo un contraste post hoc mostr'o que el percentil 1 fue evaluado significativamente 
diferente del percentil 0 (F[1-42]=5.630; p=.023) y del percentil 0,3 (F[1-42]=4.029; p=.035) mediante un test t para muestras relacionadas. 
El gr'afico de la figura \ref{fig:histogramas_juicios} muestra esas las medias para cada percentil. 

\begin{imagen}
    \file{images/exp_percentiles/marginales_percentiles.png}
    \labelname{fig:histogramas_juicios}
    \desc{Medias de los juicios de los sujetos para los 4 percentiles propuestos para la restricci'on}
    \width{9cm}
\end{imagen}

Parad'ojicamente, los sujetos consideraron mejores las melod'ias que no estaban restringidas (o que solo lo estaban moderadamente) por el modelo compuesto (Narmour-Krumhansl). 
La interacci'on entre Pieza y Percentil tambi'en result'o significativa (F[12-31]=2.937; p=.001). El gr'afico de la figura 5 permite observar que solamente la primera pieza cumple moderadamente con la predicci'on que avala la teor'ia de la implicaci'on interv'alica.


\begin{imagen}
    \file{images/exp_percentiles/marginales_por_pieza.png}
    \labelname{fig:interaccion_pieza_percentil}
    \desc{Interacci'on Pieza - Percentil.  }
    \width{9cm}
\end{imagen}


\subsubsection{Discusi\'on}
El experimento reportado en este trabajo buscaba obtener evidencia emp'irica acerca de la importancia de que la unidad relativa a la selecci'on de alturas para la 
melod'ia de un modelo generativo/computacional estuviese m'as o menos restringido por las teor'ias de implicaci'on realizaci'on -- que predice qu'e 
intervalos ser'an los m'as esperados -- y de estabilidad tonal -- que predice qu'e notas ser'an consideradas como m'as estables. 
La predicci'on dec'ia que cuanto m'as alta fuera la restricci'on, los juicios de buena forma de los oyentes ser'ian m'as altos. De este modo 
la restricci'on estipulada en el percentil 0 dar'ia lugar a juicios m'as altos que la correspondiente al percentil 1. Sin embargo, los resultados parecen 
contradecir esta predicci'on. En esta secci'on se discuten algunos puntos que pueden echar luz sobre esta contradicci'on. 

\subsubsection{En lo metodol\'ogico}
El experimento buscaba comparar la bondad de conformaci'on de las melod'ias generadas, sin embargo la tarea experimental no era de comparaci'on. Por el contrario la tarea 
era de juicio de ``buena forma'' para cada melod'ia en forma independiente. De tal manera, la perspectiva psicol'ogica de la comparaci'on no est'a capturada en esta 
metodolog'ia. Es posible pensar, entonces, que una tarea de comparaci'on en relaci'on a la ``buena forma'' podr'ia orientar al sujeto en la audici'on de los componentes 
que en efecto var'ian de una melod'ia a la otra -- en este caso restringidos por la variable independiente (percentil).

\subsubsection{En lo musicol\'ogico}
\label{sec:exp_percentiles_anal_music}
Al observar en detalle la interacci'on entre los factores Pieza y Percentil se puede apreciar que la diferencia en los juicios relativos a las melod'ias generadas de 
acuerdo a los diferentes niveles de restricci'on var'ia de acuerdo a la pieza input. Por ejemplo las melod'ias generadas a partir de la pieza 1 dieron lugar a juicios 
cuya distribuci'on es contraria a la correspondiente a los juicios para las melod'ias generadas por la pieza 5. Un an'alisis musical de las piezas compuestas por el 
programa puede entonces brindar alg'un indicio de las causas de estos resultados.

\begin{imagen}
    \file{images/exp_percentiles/fig3.png}
    \labelname{fig:percentiles_partitura1}
    \desc{Pieza input N$^{\circ}1$: Danza Alemana WoO 13 N 11 de Beethoven (arriba) y las melod'ias generadas de acuerdo a las restricciones correspondientes al percentil 0 y 
    al percentil 1 (de arriba hacia abajo) }
    \width{12cm}
\end{imagen}

La figura \ref{fig:percentiles_partitura1} muestra la partitura de la pieza input 1 y las de las melod'ias generadas conforme los percentiles 0 y 1. En lo concerniente a las relaciones interv'alicas, 
se observa que el contorno mel'odico de la melod'ia percentil 0 es m'as suavizado (con menos y menores saltos), posibilitando una mayor previsi'on de las metas mel'odicas. 
En lo relativo a la conformaci'on de la estabilidad tonal, se observa que ambas melod'ias se configuran tonalmente alrededor de si menor -- a pesar de que la pieza est'a en 
La Mayor . Mientras que la melod'ia percentil 0 se orienta hacia La Mayor a partir del comp'as 11, la melod'ia percentil 1, permanece en si menor hasta el final, de tal suerte
que el La final no puede ser interpretado como t'onica. Por el contrario el Do final de la melod'ia percentil 0 puede ser interpretado como la tercera de la funci'on t'onica 
con la correspondiente estabilidad como para garantizar un cierre y por ende mejorar la ``buena conformaci'on'': Es probable que ambos rasgos -- el mel'odico y el tonal -- est'en favoreciendo las mayores puntuaciones de los oyentes para dicha melod'ia.

\begin{imagen}
    \file{images/exp_percentiles/fig4.png}
    \labelname{fig:percentiles_partitura2}
    \desc{Pieza input N$^{\circ}5$: Danza Alemana D 974 N$^{\circ}2$ de Schubert (arriba) y las melod'ias generadas de acuerdo a las restricciones correspondientes al percentil 0 y 
    al percentil 1 (de arriba hacia abajo)}
    \width{13cm}
\end{imagen}

La figura \ref{fig:percentiles_partitura2} muestra la partitura original de la pieza input N$^{\circ}5$ con las melod'ias generadas correspondientes a los percentiles 0 y 1 respectivamente. 
En lo referente a lo interv'alico, se aprecia en la melod'ia restringida al percentil 0, un contorno mel'odico m'as suavizado que el de la otra melod'ia particularmente 
en los primeros 8 compases y sobre la cadencia final, conforme a lo predicho por la teor'ia de Narmour (en relaci'on al tama'no y la direccionalidad de los intervalos). 
En lo relativo a la configuraci'on de la tonalidad ambas melod'ias est'an claramente planteadas en Re$\flat$ Mayor (la tonalidad de la pieza input). 
Sin embargo la primera termina sobre la dominante apareciendo como inconclusa (inestable), mientras que la segunda termina sobre la t'onica, precedida por la sensible, 
de modo que aparece como m'as estable. Parece ser, entonces, que los oyentes privilegian la estabilidad tonal a la implicaci'on mel'odica a la hora de establecer sus 
juicios de ``buena conformaci'on'' (varios sujetos dejaron consignadas observaciones de esta 'indole al finalizar el test).

La naturaleza compuesta del modelo de esta unidad generativa hace que no sea posible separar claramente un componente del otro para testearlos por separado. 
Sin embargo, a la luz de estos resultados aparece como importante obtener esos datos que permitan establecer la bondad de cada componente individualmente. En tal 
sentido es posible que eliminando el componente tonal, al seleccionar input atonales, se pueda evaluar mejor la modelizaci'on del componente interv'alico.


\subsubsection{En lo computacional}
\label{sec:exp_percentiles_anal_comput}
Como se vio, del an'alisis musicol'ogico se desprende que es posible que los dos componentes del modelo requieran un tratamiento separado. 
Esto implicar'ia dise'nar unidades independientes para ellos.  Pero adem'as se observa a ra'iz de la desconfiguraci'on tonal de las melod'ias correspondientes al input 1, 
que es posible que el modelo utilizado para generar la estabilidad tonal (Krumhansl) deba ``componerse'' con el de la frase mel'odica para determinar m'as ajustadamente las 
notas de apoyo de modo de garantizar una inducci'on tonal m'as ajustada. 
Es posible tambi'en que la falla en la predicci'on se deba tambi'en a la velocidad del ritmo arm'onico. N'otese que el fragmento donde la predicci'on 
(que el percentil 0 iba a recibir un puntaje mayor) fue acertada, figura \ref{fig:percentiles_partitura1}, tiene un ritmo arm'onico que permite que se configuren unidades 
discursivas de varias notas. Sin embargo, en el fragmento correspondiente a la figura \ref{fig:percentiles_partitura2}, el ritmo arm'onico es m'as ``acelerado'' en este sentido. 
Es posible que la soluci'on a esto sea elaborar unidades discursivas que no necesariamente utilicen acordes sucesivos para los puntos de apoyo.
En todo caso, se considera necesario avanzar en la validaci'on de las decisiones te'oricas tomadas para la construcci'on del programa, sobre la base de la evidencia emp'irica 
obtenida de los juicios de los oyentes como se ha hecho en esta oportunidad.


\subsection{Comparaci\'on entre modelo de frases y modelo sin frases}
\label{sec:exp_frase}
\subsubsection{Objetivo}
Este experimento se propone poner a prueba la idea detr'as de utilizar un modelo adicional para modelar el discurso mel'odico: una melod'ia organizada
como una sucesi'on de frases deber'ia ser preferida a una organizada como una sucesi'on de notas aisladas.

\subsubsection{Sujetos}
64 personas (26 mujeres y 38 hombres) fueron convocadas voluntariamente mediante un correo electr'onico para realizar la tarea. La media de edad fue 30.6 a~nos (m'inima 17 y m'axima 59). Los sujetos fueron agrupados en tres niveles de acuerdo a su formaci'on
musical sistematica: No musicos (sin formaci'on musical; 19 sujetos), Formacion moderada (hasta 10 a~nos de formaci'on musical; 
29 sujetos), M'usicos profesionales (10 o m'as a~nos de formaci'on musical sistem'atica; 17 sujetos)

\subsubsection{Est\'imulos}
Se utilizaron los mismos fragmentos musicales que en el experimento \ref{sec:exp_percentiles}. A partir de estos se compusieron 
dos melod'ias. La primera utilizando solamente los modelos de contextos arm'onicos (secci'on \ref{sec:harmonic_contexts}) y
contornos mel'odicos (secci'on \ref{sec:melodic_contour}). La segunda, agregando a estos el modelo de las frases (secci'on 
\ref{sec:phrases}) utilizando el valor 0 para el percentil del predicado $S$ y el valor 1 para el percentil del predicado $M$.
Ambas melod'ias fueron compuestas con la misma semilla, de esta forma la r'itmica es la misma, y las diferencias entre las 
melod'ias resultantes son producto de la diferencia entre los algoritmos.

\subsubsection{Procedimiento}
Con el an'alisis de los resultados del experimento anterior se observ'o que los puntajes asignados por los sujetos a las distintas melod'ias 
fueron utilizados para realizar una comparaci'on y el valor absoluto del puntaje no fue tenido en cuenta en el an'alisis. 
Esta observaci'on dio lugar a cuestionarse si la interfaz web de la figura \ref{fig:rank_webi} era la adecuada. Dado que el objetivo de este
experimento es escencialmente comparativo, presentar las melod'ias de forma independiente y aleatoria hace que el puntaje asignado por los sujetos 
este potencialmente desvinculado en las melod'ias que se desea comparar por diversas razones. 
Adem'as, no se desea comparar todas las melod'ias puesto que en la preferencia entre dos melod'ias generadas a partir de partituras distintas entran 
en juego muchas variables que no se desean considerar en este an'alisis.
De esta forma se dise~no una nueva interfaz, exhibida en la figura \ref{fig:comparision_webi}, con el objeto de que permita comparar entre las melod'ias 
que se desea comparar, y solo entre ellas. Para escuchar las melod'ias hay que hacer click en cada uno de los parlantes, y haciendo click en los c'irculos
de la recta es posible expresar una preferencia de forma gradual: el centro expresa que no se perciben diferencias entre las melod'ias, y los cuatro c'irculos
para cada costado permiten expresar niveles de preferencia. Por ejemplo, en la figura \ref{fig:comparision_webi} se est'a expresando una preferencia 
gradual hacia la melod'ia de la izquierda.

\begin{imagen}
    \file{images/screenshot_experimento_comparacion.png}
    \labelname{fig:comparision_webi}
    \desc{Interfaz web utilizada para comparar entre melod'ias. Haciendo click en cada parlante es posible escuchar una melod'ia.
    La preferencia se establece llevando el selector violeta hacia el parlante de la melod'ia preferida. En el caso de esta 
    figura, se est'a estableciendo una preferencia parcial hacia la canci'on de la izquierda.}
    \width{13cm}
\end{imagen}

Utilizando esta interfaz, se les present'o a los sujetos 5 pares de melod'ias, aleatorizando el orden de cada par de melod'ias.
A su vez, cada par de melod'ias era aleatorizado para evitar tendencias (izquierda sobre derecha). 
Una vez expresada la preferencia y habiendo escuchado las melod'ias, es posible expresar la preferencia del siguiente par de 
melod'ias. Al finalizar el experimento los sujetos respond'ian un cuestionario de datos personales.

\subsubsection{Resultados}
A modo exploratorio se ralizaron los gr'aficos de la figura \ref{fig:comparision_distrs}. En cada uno de estos gr'aficos, el eje X 
corresponde al selector de la figura \ref{fig:comparision_webi}, llevando a la derecha las melod'ias que fueron generadas
con el algoritmo de frases. En los gr'aficos se puede observar una leve tendencia, hacia la derecha. 

Esta leve tendencia hacia la derecha impacta en que las medias de los juicios de preferencia sean mayormente positivas como se exhibe en la tabla 
\ref{fig:comparision_means}. El 'unico caso en donde la media no es positiva es en la pieza n'umero 5, que es la misma que dio al rev'es de lo 
esperado en el experimento anterior (ver \ref{sec:exp_percentiles_anal_music})

\begin{figure}[htp]
    \begin{center}
        \begin{tabular}{cc}
        Pieza & Media \\
        \hline 
        1		&		0.277   \\
        2		&		0.554   \\
        3		&		0.185   \\
        4		&		0.156   \\
        5		&		-0.231   \\


        \end{tabular}
        \caption{Valores medios para cada preferencia}
        \label{fig:comparision_means}
    \end{center}      
\end{figure}


Si bien las medias son en su mayor'ia positivas, los valores son bastante cercanos a 0. Analizando estos datos m'as rigurosamente, se los someti'o a 
una prueba t para analizar si estas medias eran significativamente diferentes de 0.
El resultado, lamentablemente arroj'o lo que se esperaba a partir de observar las medias: la probabilidad de que estas sean distintas de cero es baja, salvo
en la pieza 2 que es del 92\%.

\begin{figure}[htp]
    \begin{flushleft}
        \begin{tabular}{cc}
        \includegraphics[width=7.5cm]{images/exp_comparision/beet_wo13.png} &
        \includegraphics[width=7.5cm]{images/exp_comparision/beet_wo14_2.png} \\
        Pieza n'umero 1 & Pieza n'umero 2 \\ 
        \includegraphics[width=7.5cm]{images/exp_comparision/beet_wo14_8.png} &
        \includegraphics[width=7.5cm]{images/exp_comparision/schub_d973.png} \\
        Pieza n'umero 3 & Pieza n'umero 4 \\ 
        \includegraphics[width=7.5cm]{images/exp_comparision/schub_d974.png} & \\
        Pieza n'umero 5 & \\ 
    %%	\vspace{1cm} & \\

        \end{tabular}
        \caption{Juicios de preferencia para cada par de melod'ias. El eje X representa el selector de la figura \ref{fig:comparision_webi}, 
        llevando a la derecha las melod'ias que fueron compuestas utilizando el algoritmo de frases, y a la derecha las que no. Siendo asi,
        mayor cantidad de valores a la derecha representa una preferencia de melod'ias generadas con el algoritmo para las frases y viceversa.}
        \label{fig:comparision_distrs}
    \end{flushleft}      
\end{figure}

\subsubsection{Discusi\'on}
En un primer momento estos resultados se revelaron desalentadores puesto que se cre'ia que iba a observar una tendencia m'as marcada y estad'isticamente
significativa. De todas formas, el hecho de que se manifestara una tendencia parcial dio a lugar a rever el procedimiento con el que se realiz'o el 
experimento. Con este objetivo, se volvi'o a relevar el experimento con mucho cuidado a escuchar las diferencias entre las melod'ias. 

Lo primero que se observ'o fue que para evaluar qu'e melod'ia era ``mejor'' era necesario reproducirla en el contexto de la partitura original, eliminando, de existir, la melod'ia original de la partitura. Es importante notar la diferencia que existe entre el juicio de preferencia cuando se refiere
a la melod'ia aislada respecto a la melod'ia en el contexto de la partitura. En el primer caso corresponde a un juicio de ``buena forma'' de la 
melod'ia como ente musical aislado, y en el segundo corresponde a un juicio de ``buen ajuste'' de la melod'ia a la partitura original. Siendo que 
los modelos propuestos en este trabajo se entrenan con aspectos de la partitura en s'i, desde la tonalidad y los contornos mel'odicos hasta el ritmo
arm'onico, es de esperarse que la evaluaci'on sea en el contexto de la partitura original.

Las melod'ias en el contexto de la pieza original permitieron distinguir en ciertos momentos d'onde se encontraba el algoritmo de frases y d'onde no.
No obstante la diferencia segu'ia siendo sut'il. Esto en parte se debe a que las melod'ias que fueron construidas sin el algoritmo para frases
tienen en cuenta tambi'en el ritmo arm'onico, puesto que el \emph{pitch profile} se actualiza con cada acorde (ver secci'on \ref{sec:harmonic_context_model}).
Esto hace que sea razonable que las las melod'ias generadas por ambos modelos difieran s'olo de una forma sutil.

Respecto a la forma de presentar los est'imulos, se observ'o que el final de las melod'ias era frecuentemente utilizado por los sujetos para
efectuar su juicio. Dado que no se construy'o un modelo espec'ifico para los finales de las melod'ias, muchas veces estas terminan inconclusas afectando fuertemente los juicios de valor.
Estos finales inconclusos son producto del azar, y no se los desea evaluar. Una forma para evitarlos es disminuir gradualmente el volumen de la pieza conjunta
(partitura original y melod'ia) de forma tal que se entienda que ese no es un final, y no sea evaluado como tal.


\subsection{Experimento contra una l\'inea de base}
\label{sec:exp_baseline}
\subsubsection{Objetivo}
Siendo que el objetivo de este trabajo se basa en construir modelos generativos con fundamentos cognitiva para luego poder medir caracter'isticas puntuales
en la cognici'on musical, un primer paso en pos de este objetivo (y a su vez el objetivo de este experimento) es mostrar que los modelos propuestos 
en este trabajo capturan en cierto punto parte del proceso cognitivo realizado por un oyente al escuchar una pieza musical. 

\subsubsection{Sujetos}
79 personas (16 mujeres y 69 hombres\footnote{La diferencia entre la cantidad de hombres y mujeres se debe a que la mayor'ia de los sujetos fueron convocados a partir de un mail enviado a la lista de alumnos del departamento de computaci'on}) fueron convocadas voluntariamente mediante un correo electr'onico para realizar la tarea. La media de edad fue 29.6 a~nos (m'inima 19 y m'axima 60). Dado que en este caso hab'ia muy pocos sujetos cuya formaci'on
superara los 6 a~nos (18 sujetos), en este caso consider'o como posibles agrupacioines a las siguientes: No m'usicos (sin formaci'on musical; 39 sujetos)
, con al menos 1 a~no de estudio (40 sujetos), con al menos 3 a~nos de estudio (24 sujetos) y con al menos 5 a~nos de estudio (19 sujetos). 

\subsubsection{Est\'imulos}
Se eligieron 5 fragmentos musicales del corpus Melisma\footnote{\url{http://www.link.cs.cmu.edu/music-analysis/ftp-contents.html}}:
(1) Beethoven, Sonata Op. 13; (2) Brahms, ``Und gehst du ueber den Kirchhof''; (3) Chopin, Mazurka Op. 63, No. 2; 
(4) Mozart, ``Eine Kleine Nachtmusik''; y (5) Schumann, ``Aus meinen Thranen spriessen''. Para cada uno de estos fragmentos se generaron dos 
melod'ias. La primera utilizando solamente el modelo de la r'itmica (secci'on \ref{sec:metric_model}) y un 'unico pitch profile global sin hacer 
detecci'on de acordes. La segunda, utilizando todos los modelos en conjunto: r'itmica, pitch profile con detecci'on de acordes, contornos 
mel'odicos, algoritmo para el fraseo y algoritmo para la elaboraci'on mot'ivica. Ambas melod'ias fueron generadas teniendo en cuenta lo elaborado en
el experimento anterior.

\subsubsection{Procedimiento}
El procedimiento fue exactamente igual al anterior. La diferencia ahora radica en el significado que tiene el juicio de preferencia, 
siendo que los est'imulos fueron presentados junto a la partitura original. Se elimin'o el final de las piezas para evitar las tendencias
que podr'ia ocacionar.

\subsubsection{Resultados}
Al igual que en el experimento anterior, se construyeron los gr'aficos de la figura \ref{fig:baseline_distrs}, que representan el 
juicio de preferencia de los sujetos. Estos gr'aficos muestran una tendencia mucho m'as marcada que los del experimento anterior.
 Un test t comparando las medias contra 0 revela que, exceptuando la pieza 3 
cuyo p-valor es de 0.014, los juicios de valor son significativamente distintos de 0 (p-valores del orden de $10^{-5}$). En los gr'aficos de la figura \ref{fig:baseline_pvalues} se puede ver el p-valor de una prueba t para distintos valores a comparar. Se grafica una l'inea de color verde en el valor de 0.01, de esta forma se puede saber en que punto se empieza a rechazar la hip'otesis de que esa sea la media. Adem'as el pico, marcado con una l'inea
vertical roja indica la media, que coincide con el valor m'as probable de la prueba t.

Un resultado por dem'as curioso surgi'o de la comparaci'on entre los juicios de preferencia entre m'usicos y no m'usicos. Dado que en la poblaci'on 
de sujetos obtenida no hab'ia una gran cantidad de sujetos m'usicos con m'as de 3 y 5 a~nos de experiencia, hubo que incurrir en un proceso de muestreo
para poder realizar la comparaci'on. El proceso de muestreo consiste en tomar sucesivas muestras de la poblaci'on de no m'usicos del mismo tama~no que la 
poblaci'on de m'usicos con la que se desea comparar. Teniendo las sucesivas muestras de los no m'usicos, se procede a comparar las medias: la media de los
m'usicos con los datos de los no m'usicos y viceversa. Los resultados de este proceso arrojaron, en algunos fragmentos musicales, que no existen diferencias
significativas entre los juicios, sin embargo, hay dos fragmentos donde parece haber una diferencia. En la figura \ref{fig:pvalues_musician_vs_nonmusician}
se exhiben todos los p-valores arrojados por realizar un proceso de 100 muestras. Observar que hay 200 valores, puesto que se realizan dos pruebas t por 
cada muestra. A modo de comparaci'on, en la figura \ref{fig:pvalues_arbitrary_partition} se exhibe el mismo proceso de muestreo partiendo el conjunto 
de sujetos en dos subconjuntos de tama~no 24 sin ningun criterio en particular salvo que estos sean disjuntos en los sujetos que contienen.

\subsubsection{Discusi\'on}
La tendencia, estad'isticamente significativa, encontrada en los juicios de preferencia corrobora la hip'otesis de que los modelos de detecci'on de 
acordes, contornos mel'odicos, frases y elaboraciones mot'ivicas, en suma, capturan en cierto punto parte del proceso cognitivo que realiza un oyente al
escuchar una base de piano. Esta conclusi'on se desprende de la interpretaci'on previamente explica del juicio de preferencia de una melod'ia sobre
una base de piano. Este juicio, interpretado como de ``buen ajuste'', estar'ia indicando que es m'as ``ajustada'' una melod'ia que siga los modelos 
propuestos, sobre que no los siga.

Adem'as de este resultado, que le da sustento al trabajo completo, se encontr'o informaci'on que permitir'ia en el futuro establecer alg'un criterio, 
en t'erminos psicol'ogicos, que permita discriminar entre personas entrenadas musicalmente y no a partir de nuevos experimentos cuyo objetivo sea
ese, y del an'alisis musicol'ogico de las partituras buscando diferencias entre las que tuvieron tendencia y las que no. 


\begin{figure}[htp]
    \begin{flushleft}
        \begin{tabular}{cc}
        \includegraphics[width=7.5cm]{images/exp_baseline/all_data/beet-son13-ii.png} &
        \includegraphics[width=7.5cm]{images/exp_baseline/all_data/brahms-undgehst.png} \\
        \footnotesize{ Pieza n'umero 1} & \footnotesize{Pieza n'umero 2} \\ 
        \includegraphics[width=7.5cm]{images/exp_baseline/all_data/chop-maz63-2.png} &
        \includegraphics[width=7.5cm]{images/exp_baseline/all_data/mzt-ekn-ii.png} \\
        \footnotesize{Pieza n'umero 3} & \footnotesize{Pieza n'umero 4} \\ 
        \includegraphics[width=7.5cm]{images/exp_baseline/all_data/schum-thranen.png} & \\
        \footnotesize{Pieza n'umero 5} & \\ 
    %%	\vspace{1cm} & \\

        \end{tabular}
        \caption{Juicios de preferencia para cada par de melod'ias. El eje X representa el selector de la figura \ref{fig:comparision_webi}, 
        llevando a la derecha las melod'ias que fueron compuestas utilizando el algoritmo de frases, y a la derecha las que no. Siendo asi,
        mayor cantidad de valores a la derecha representa una preferencia de melod'ias generadas con el algoritmo para las frases y viceversa.}
        \label{fig:baseline_distrs}
    \end{flushleft}      
\end{figure}

\begin{figure}[htp]
    \begin{flushleft}
        \begin{tabular}{cc}
        \includegraphics[width=7.5cm]{images/exp_baseline/all_data/pvalues/beet-son13-ii.png} &
        \includegraphics[width=7.5cm]{images/exp_baseline/all_data/pvalues/brahms-undgehst.png} \\
        \footnotesize{Pieza n'umero 1} & \footnotesize{Pieza n'umero 2} \\ 
        \includegraphics[width=7.5cm]{images/exp_baseline/all_data/pvalues/chop-maz63-2.png} &
        \includegraphics[width=7.5cm]{images/exp_baseline/all_data/pvalues/mzt-ekn-ii.png} \\
        \footnotesize{Pieza n'umero 3} & \footnotesize{Pieza n'umero 4} \\ 
        \includegraphics[width=7.5cm]{images/exp_baseline/all_data/pvalues/schum-thranen.png} & \\
        \footnotesize{Pieza n'umero 5} & \\ 
    %%	\vspace{1cm} & \\

        \end{tabular}
        \caption{p-valores para distintos valores de la media. La recta horizontal de color verde indica el p-valor 0.01 y la recta roja vertical indica
        el valor m'as probable, que corresponde a la media}
        \label{fig:baseline_pvalues}
    \end{flushleft}      
\end{figure}

\begin{imagen}
    \file{images/exp_baseline/musician_vs_nonmusician/schum-thranen.png}
    \labelname{fig:pvalues_musician_vs_nonmusician}
    \desc{Histograma de p-valores arrojados por el proceso de muestreo para la pieza 5 comparando entre m'usicos y no m'usicos}
    \width{8cm}
\end{imagen}

\begin{imagen}
    \file{images/exp_baseline/arbitrary_partition/schum-thranen.png}
    \labelname{fig:pvalues_arbitrary_partition}
    \desc{Histograma de p-valores arrojados por el proceso de muestreo para la pieza 5, comparando muestras arbitrarias de tama~no 24 con la 'unica 
    condici'on de que sean disjuntas en los sujetos}
    \width{8cm}
\end{imagen}


\chapter{Conclusiones}
En el presente trabajo se plantea como hip'otesis que la experimentaci'on con m'usica generada autom'aticamente a partir de 
algoritmos con fundamentaci'on cognitiva, podr'ia permitir validar las teor'ias en las que estos algoritmos fueron fundados. 
Dado que esta hip'otesis no ha sido planteada con anterioridad, y por consiguiente no se han construido modelos generativos 
teniendo en cuenta todas las teor'ias que se consideran pertinentes, gran parte del trabajo consisti'o en construirlos. 
Se opt'o por un enfoque estad'istico para construir los modelos, por sobre enfoques l'ogicos o basados en reglas, puesto 
que este es m'as flexible, y la noci'on de ``entrenar el modelo'' pasa a tener una interpretaci'on intuitiva. 

Es importante antes de embarcarse en semejante tarea, comprender cual es su alcance para poder definir un trabajo que se mantenga 
acotado. Es por esto que se decidi'o a priori que los modelos propuestos ser'ian entrenados con una s'ola partitura evitando
el problema de tener que combinar informaci'on de distintas partituras. Adem'as se decidi'o trabajar solamente con l'ineas 
mel'odicas y no generar la estructura de la pieza.
Teniendo en cuenta estas restricci'ones, se aport'o con el desarrollo de los siguientes modelos:

\begin{itemize}
 \item Modelo para la r\'itmica: Basado en la definici'on de \cite{LerdahlJackendoff83} de acento m'etrico, las nociones de 
 clase de equivalencia por \cite{Benjamin84} y \cite{clarke1987categorical} y la noci'on de pulso definida por \cite{snyder2001}
 se construy'o un modelo que genera r'itmicas ``en relaci'on a la partitura de entrenamiento''

 \item Modelo para los contornos mel'odicos: Basado en la teor'ia de Implicaci'on-Realizaci'on de \cite{Narmour90}, 
 se propone un modelo simple que permite cuantificar el grado de implicaci'on que genera un intervalo mel'odico sobre el siguiente.

 \item Modelo para contextos arm'onicos: Basado en los trabajos de \cite{Krumhansl90} y \cite{Lerdahl2001} se propone un 
 modelo basado en un marco bayesiano para cuantificar la estabilidad de una nota en funci'on de la tonalidad inferida del tema y 
 el acorde que gobierna la sonoridad del momento.

 \item Modelo para frases mel'odicas: Utilizando los modelos anteriores, y basado en las definiciones para frase musical encontradas
 en \cite{rothstein1989phrase}, se construye un modelo que organize la melod'ia en t'erminos de unidades discursivas.

 \item Modelos para generar relaciones mot'ivicas: Se propone utilizar el modelo de los Restaurantes Chinos \citep{Teh2007} 
 para generar repetici'on de partes, siguiendo la teor'ia de extracci'on de pistas presentada en \cite{Deliege90}, de forma tal 
 que la cantidad de repeticiones de una parte no sea excesiva ni nula.

\end{itemize}

Utilizando estos modelos se procedi'o a validar su verosimilitud en t'erminos cognitivos. Plantearlo result'o sencillo, sin embargo
llevarlo a la pr'actica no lo fue: en los modelos que se proponen, hay una gran cantidad de variables en juego, y al ser este 
enfoque novedoso, no hab'ia en la literatura experimentos similares en los cuales basarse. Sumado a esto 'ultimo, las decisiones
que implican en el del dise~no del experimento: qu'e piezas utilizar, qu'e instrumento se le asignar'a a la melod'ia generada, 
si la melod'ia se presentar'ia sola, o junto a la pieza original, si el experimento ser'a presencial o no, la construcci'on 
interfaz para presentar los est'imulos, etc.  Todo esto impact'o en que los primeros experimentos (secciones 
\ref{sec:exp_percentiles} y \ref{sec:exp_frase}) no mostraran un comportamiento estad'isticamente significativo. Sin embargo
sirvieron como aprendizaje para el 'ultimo, en donde se pudo validar el supuesto de este trabajo. 

En el experimento de la secci'on \ref{sec:exp_baseline} se obtuvieron dos resultados importantes. El primero, 
objetivo del experimento, mostr'o que los modelos definidos en este trabajo modelan en cierto punto la forma en la que 
un oyente estructura su percepci'on al escuchar una pieza musical. El segundo, resultado no buscado, mostr'o que ante los
mismos est'imulos, los sujetos con entrenamiento musical se comportaban significativamente diferente de los sujetos sin
entrenamiento musical. Esto indica que en el futuro ser'ia posible identificar diferencias perceptuales entre ellos en t'erminos 
de los modelos planteados en este trabajo. Tambi'en es importante trabajar con ciertas simplificaciones en los modelos propuestos:
\begin{itemize}
 \item Mejorar la detecci'on de los cambios de foco arm'onico de forma tal de inferirlo sin necesidad de que est'e expl'icito
 como un acorde tocado por el piano.

 \item Incorporar nuevas variables para el modelo de la jerarqu'ia tonal. El modelo propuesto por \cite{Krumhansl90} permite 
 describir la tonalidad en la que se encuentra una pieza, pero como se muestra en el experimento \ref{sec:exp_percentiles} 
 variables como el acento m'etrico y la duraci'on de la nota a generar son importantes a la hora de construir una melod'ia que 
 haga que el oyente induzca la tonalidad que se est'a modelando.

 \item Eliminar la simplificaci'on utilizada por el modelo de Narmour en donde se asume que todos los intervalos son de tipo
 implicativo. Manejar los puntos de cierre permitir'ia adem'as una mejor definici'on operativa de frase mel'odica.

 \item Incorporar el uso de silencios.

\end{itemize}


De esta forma se concluye que la metodolog'ia propuesta, no s'olo permitir'ia validar teor'ias cognitivas para la m'usica, sino
que podr'ia aportar nueva informaci'on relevante al area. 
