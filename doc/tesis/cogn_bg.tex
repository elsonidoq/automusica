\section{Background cognitivo}
En esta secci'on se describir\'an ciertas cuestiones relacionadas con teor\'ia musical y con el estudio
de la percepci\'on de la misma. 

\red{En este parrafo tenia ganas de separar un poco el tiempo de la altura para despues poder expluicar mejor, pero no me gusta como quedo}

Principalmente, la m'usica se descompone en dos grandes dimensiones: la del \texttt{tiempo} y la de la 
\texttt{altura}. La dimensi\'on del tiempo se refiere a las duraci\'ones de los distintos eventos que 
ocurren en una pieza musical. La dimensi\'on de la altura, por otro lado, se refiere a la percepci\'on de las distintas notas que ocurren 
en un tema. 

Si bien estas dos dimensiones se presentan por separado, de ninguna manera son independientes, aunque muchas veces se asume un cierto
grado de independencia para facilitar el an'alisis. 

\subsection{El concepto de \emph{beat}}

\subsection{Acentuaci\'on}
Una car'acteristica de la m'usica, tambien compartida con el habla, es que un mismo evento no es percibido de la misma forma seg'un
el contexto en el que ocurre. Hay varios factores que afectan el contexto, y uno de ellos es la \texttt{acentuaci\'on}. 

Un evento musical es escuchado como acentuado si es enfatizado de alg\'una forma. Lerdahl y Jackendoff (\cita) distinguen tres tipos de 
acentos: los acentos fenomenales \red{se que no es esta la traducci\'on, como se dice en espa~nol?}, estructurales y m'etricos.
Un acento fenomenal es cualquier evento que de 'enfasis o estress a un momento en la pieza musical. \red{Que ejemplos doy? esto lo va a 
leer gente que \emph{no sabe} musica}. Los acentos estructurales son puntos de apoyo para finalizar una parte o una frase. 
Por 'ultimo, los acentos m'etricos son aquellos beats relativamente fuertes dentro del contexto m'etrico donde suceden.\newline
\red{tengo que explicar que es un beat?} \red{que es el contexto metrico? no se como explicarlo, para mi es como el compas jaja}

Es importante notar que esta categorizaci'on no es excluyente, es decir, un beat puede estar en m'as de una de las categor'ias mencionadas.
Para ejemplificar, los acentos fenomenales act'uan como pistas que ayudan al oyente a extrapolar un patr'on regular de acentos m'etricos, 
de esta forma, beats que son acentuados con un acento fenomenal, tambi'en lo son con un acento m'etrico.


