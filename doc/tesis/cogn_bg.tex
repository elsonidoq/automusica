%\section{Fundamentos cognitivos y musicales}
%\label{sec_cogn_bg}
%Principalmente, la m'usica tiene dos grandes componentes: la del \texttt{tiempo} y la de la 
%\texttt{altura}. La dimensi\'on del tiempo se refiere a las duraci\'ones de los distintos eventos que 
%ocurren en una pieza musical. Por otro lado, la altura se refiere a la percepci\'on de las distintas notas que ocurren 
%en un tema.es Es importante notar la diferencia entre la altura de una nota y la frecuencia de la misma; la altura refiere
%a la percepci'on de una frecuencia.
%
%Si bien estas dos dimensiones se presentan por separado, de ninguna manera son independientes. Ser'an abordadas de esta forma 
%para facilitar una primera descripci'on. De ser necesario, posteriormente se elaborar'an los conceptos que refieran a las interacciones entre 
%estas dos componentes. 

%\subsection{La tensi\'on y lo esperable}
%\label{subsec_tension}
%\begin{itemize}
%  \item 
%
%\end{itemize}
%\comment{aca voy a hablar sobre la relacion que hay entre la tension y lo esperable, definiendo como esperable a un evento en un cierto contexto, luego voy a distinguir entre
%dos tipos de contextos: el horizontal y el vertical. El contexto horizontal tiene que ver con patrones en el tiempo, y el vertical con relaci'ones armonicas en un cierto
%instante. Esto despues lo voy a usar para los modelos de la ritmica (solo usa contexto horizontal) y de la melodia (usa los dos contextos y arma un modelo conjunto)}


%
%\subsection{Acentuaci\'on}
%Una car'acteristica de la m'usica, tambi'en compartida con el habla, es que un mismo evento no es percibido de la misma forma seg'un
%el contexto en el que ocurre. Hay varios factores que afectan el contexto, y uno de ellos es la \texttt{acentuaci\'on}. 
%
%Un evento musical es escuchado como acentuado si es enfatizado de alg\'una forma. Lerdahl y Jackendoff\cite{LerdahlJackendoff83} distinguen tres tipos de 
%acentos: los acentos fenom'enico, estructurales y m'etricos.
%Un acento fenomenal es cualquier evento que de 'enfasis o estress a un momento en la pieza musical. \alert{Que ejemplos doy? esto lo va a 
%leer gente que \emph{no sabe} musica}. Los acentos estructurales son puntos de apoyo para finalizar una parte o una frase. 
%Por 'ultimo, los acentos m'etricos son aquellos beats relativamente fuertes dentro del contexto m'etrico donde suceden.
%
%Kramer\cite{Kramer88} tambi'en reconoce tres tipos distintos de acentos, llamados \emph{m'etrico}, \emph{stress} y \emph{r'itmico}. Esta categorizaci'on
%es equivalente a la dada por Lerdahl y Jackendoff.
%
%Meyer\cite{CooperMeyer60} no elabora\footnote{checkear que sea realmente asi} una taxonom'ia de acentos. Su inter'es no est'a en saber \emph{qu'e} hace
%que un pulso est'e acentuado sino \emph{c'omo} opera un pulso acentuado en terminos de estructura r'itmica. En estos t'erminos distingue
%que los pulsos acentuados son de alguna forma el punto de foco donde los beats circundantes se agrupan. 
%
%Es importante notar que este tipo de categorizaciones no son excluyentes, es decir, un beat puede tener m'as de un tipo de acento al mismo tiempo.
%Para ejemplificar, dentro de la teor'ia de Lerdahl y Jackendoff se describe el proceso cognitivo por el cual se infieren los acentos m'etricos. Este
%proceso consiste en que el oyente utiliza los acentos fenomenales y estructurales como pistas para extrapolar un patr'on regular de acentos m'etricos. 
%Dentro de 'este contexto, no es la excepci'on, sino la regla que haya beats que son acentuados con un acento fenomenal y m'etrico al mismo tiempo.
%
%
