\section{El modelo compuesto}
Utilizando los modelos definidos en las secciones \ref{sec:metric_model}, \ref{sec:harmonic_contexts} y \ref{sec:melodic_contour}, es posible definir
un proceso generativo para componer lineas mel'odicas de forma relativamente sencilla. El algoritmo se definir'a en pseudoc'odigo para que quede en una representaci'on 
compacta, sin embargo se deduce de los roles que se le asign'o a cada modelo: para cada nota a generar, el modelo de la r'itmica eligir'a su duraci'on, 
mientras que el modelo de Narmour elijir'a el tipo de intervalo que deber'a tocarse en funci'on del 'ultimo intervalo tocado, y por 'ultimo el modelo del contexto arm'onico
elijir'a una nota dentro de las que forman el intervalo determinado por el modelo de Narmour de acuerdo a la tonalidad inferida.

\begin{algoritmo}
create_melody(rhythm_model, harmonic_context, contour_model, duration, available_notes)
    notes := []
    while last_note.end < duration
        new_duration := rythm_model.next_duration()
        new_interval_type := contour_model.next_interval_type()

        candidate_notes := [n for n in available_notes if interval_type(last_note, n) = new_interval_type]
        new_pitch := harmonic_context.pick_note(candidate_notes)

        last_note= Note(last_note.start, new_duration, new_pitch)
        notes.append(last_note) 

    return notes
\end{algoritmo}
\subsection{Problemas}
\comment{aca la idea es explicar problemas que tienen estos modelos en terminos musicales. Basicamente voy a hablar del rol que juega la repetici\'on en nuestra
escucha musical y en la construccion de motivos. La idea es motivar alguna forma de generar repeticiones parciales, o elaboraciones motivicas. Esto motivaria la seccion que viene}
