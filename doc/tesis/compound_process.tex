\section{El modelo compuesto}
Utilizando los modelos definidos en las secciones \ref{sec:metric_model}, \ref{sec:harmonic_contexts} y \ref{sec:melodic_contour}, es posible definir
un proceso generativo para componer lineas mel'odicas de forma relativamente sencilla. 
%Tener en cuenta que dependiendo de la elecci'on del modelo de contorno mel'odico, la cadena de Narmour o la distribuci'on de Narmour, el algoritmo se ver'a
%afectado. El primer modelo permite elegir un conjunto de notas candidatas, mientras que el segundo permite incorporar dentro de una distribuci'on de 
%probabilidades el modelo de contornos melodicos con el modelo del contexto arm'onico. 
A continuaci'on se definir'a el algoritmo exhibiendo el pseudoc'odigo para que quede en una representaci'on compacta, 
sin embargo se deduce de los roles que se le asign'o a cada modelo: 
para cada nota a generar, el modelo de la r'itmica eligir'a su duraci'on, mientras que el modelo de Narmour en conjunto con el de contexto arm'onico 
elijir'an su altura tocarse en funci'on del 'ultimo intervalo tocado.

\begin{algoritmo}
create_melody(rhythm_model, harmonic_context, contour_model, 
              total_duration, available_notes, context)
    notes := []
    while last_note.end < total_duration
        new_duration := rythm_model.next_duration()

        pitch_distribution= {}
        for note in available_notes
            prob := harmonic_context.get_prob(note)*contour_model.get_prob(note, context)
            pitch_distribution[note] := prob 
        new_pitch := pick(pitch_distribution)

        last_note := Note(last_note.start, new_duration, new_pitch)
        notes.append(last_note) 

        n1, n2 := context
        context := (n2, last_note)

    return notes
\end{algoritmo}

N'otese que el diccionario pitch\_distribution no necesariamente suma 1, sin embargo no es necesario que sume uno, puesto que lo que se desea es elejir un elemento con
probabildad proporcional al valor que este le atribuye, que es lo que hace la funci'on \emph{pick}.

De esta forma se puede expresar en t'erminos probabil'isticos la elecci'on total de cada nota. Sean $d_j$ la sucesi'on de duraci'ones y $n_j$ 
la sucesi'on de notas, y $\theta$ el vector de par'ametros del contexto arm'onico, entonces la probabilidad de una nota estar'a dada por:

$$P(d_j, n_j | d_{j-1}, n_{j-1}, n_{j-2}) = P(d_j|d_{j-1})P(F_1(n_{j-2}, n_{j-1}, n_j), \cdots, F_k(n_{j-2}, n_{j-1}, n_j)) P(n_j | \theta)$$

