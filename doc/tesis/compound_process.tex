\section{El modelo compuesto}
Utilizando los modelos definidos en las secciones \ref{sec:metric_model}, \ref{sec:harmonic_contexts} y \ref{sec:melodic_contour}, es posible definir
un proceso generativo para componer lineas mel'odicas de forma relativamente sencilla. 
Tener en cuenta que dependiendo de la elecci'on del modelo de contorno mel'odico, la cadena de Narmour o la distribuci'on de Narmour, el algoritmo se ver'a
afectado. El primer modelo permite elegir un conjunto de notas candidatas, mientras que el segundo permite incorporar dentro de una distribuci'on de 
probabilidades el modelo de contornos melodicos con el modelo del contexto arm'onico. 
A continuaci'on se definir'a el algoritmo utilizando el modelo de Markov para elegir los intervalos mel'odicos, se exhibira el pseudoc'odigo 
para que quede en una representaci'on compacta, sin embargo se deduce de los roles que se le asign'o a cada modelo: 
para cada nota a generar, el modelo de la r'itmica eligir'a su duraci'on, mientras que el modelo de Narmour elijir'a el tipo de intervalo que deber'a 
tocarse en funci'on del 'ultimo intervalo tocado, y por 'ultimo el modelo del contexto arm'onico elijir'a una nota dentro de las que forman 
el intervalo determinado por el modelo de Narmour de acuerdo a la tonalidad inferida.

\begin{algoritmo}
create_melody(rhythm_model, harmonic_context, contour_model, duration, available_notes)
    notes := []
    while last_note.end < duration
        new_duration := rythm_model.next_duration()
        new_interval_type := contour_model.next_interval_type()

        candidate_notes := []
        for candidate_note in available_notes
            if interval_type(last_note, n) = new_interval_type
                candidate_notes.append(candidate_note)

        new_pitch := harmonic_context.pick_note(candidate_notes)

        last_note := Note(last_note.start, new_duration, new_pitch)
        notes.append(last_note) 

    return notes
\end{algoritmo}

Hay una diferencia entre los dos modelos de Narmour que es importante aclarar. Observar que el algoritmo elije primero un intervalo mel'odico, y luego
elije una nota restringida a que cumpla con ese intervalo mel'odico. Es decir, si se definen variables aleatorias $d_j, i_j, n_j$ para la duraci'on, 
el tipo de intervalo mel'odico con la codificaci'on definida en \ref{sec:markov_narmour}, y la altura j-'esimas, la probabilidad de tocar una cierta nota estar'a definida de la siguiente forma:

$$P(d_j, n_j | d_{j-1}, i_{j-1}) = P(d_j|d_{j-1})P(i_j| i_{j-1})P(n_j | i_j, \theta)$$

Es decir, se esta condicionando la elecci'on de la altura al tipo de intervalo mel'odico, cuando al utilizar el otro\alert{ponerle nombre} modelo,
la variable $i$ desaparece y la expresi'on quedar'a:

$$P(d_j, n_j | d_{j-1}, n_{j-1}, n_{j-2}) = P(d_j|d_{j-1})P(F_1(n_{j-2}, n_{j-1}, n_j), \cdots, F_k(n_{j-2}, n_{j-1}, n_j)) P(n_j | \theta)$$

\section{Sobre los silencios}
