\section{Notas, alturas, acordes y tonalidades}
\label{sec_cogn_bg}
Esta secci'on cumple dos objetivos dentro de este trabajo; por un lado, sirve para el lector no familiarizado con teor'ia musical para establecer
ciertos conceptos generales que son de gran importancia en el desarrollo de este trabajo. Por otro lado, permitir'a ademas acentar un glosario
que se utilizar'a a lo largo de las diferentes secciones.
Principalmente, la m'usica se divide en dos grandes componentes u ejes: la del \texttt{tiempo} y la de la \texttt{altura}. 

La dimensi'on temporal responde a \emph{cu'ando} ocurre un evento, mientras que la altura expresa caracter'isticas sobre \emph{qu'e}
tipo de evento es.  Dejando fuera del an'alisis a los instrumentos de percusi'on, la altura es refiere a la percepci'on de un cierto sonido. 
Esta percepci'on si bien esta relacionada con caracter'isticas intr'insecas del sonido, como su intensidad, timbre y frecuencia, 
tambi'en hace referencia
al proceso cognitivo que ocurre por parte del oyente. De esta forma, si bien un sonido de una frecuencia de $440Hz$ es percibido como una nota 
$La$, tambi'en lo es un sonido de frecuencia $441Hz$.

Dentro de la m'usica occidental, existen criterios dentro de ambas componentes que ayudan a la organizaci'on de una pieza musical. 
En lo subsiguiente se dar'a cierto conocimiento general sobre estos criterios, introduciendo ciertos conceptos de la teor'ia musical, puesto que son necesarios
para abordar los fundamentos cognitivos y los modelos propuestos. 

Por 'ultimo, dado que a lo largo del presente trabajo se exhiben ejemplos, es necesario que el lector sepa comprender una partitura sencilla, de forma que se se 
explicar'a brevemente como ciertos conceptos introducidos se notan en el lenguaje musical.

\subsection{Sobre la organizaci\'on temporal}
El primer concepto importante a tener en cuenta, es que el tiempo es \emph{discreto}. Esto no quiere decir que a la hora de la producci'on musical lo sea, sino que 
en la abstracci'on musical, existe una unidad de referencia temporal, a la que se llama \emph{beat}. En torno a ella se organiza la estructura temporal, tocando
siempre notas que duren una fracci'on del mismo. De esta forma, para poder reproducir una partitura es necesario saber la duraci'on de un \emph{beat} (denominado
\emph{tempo}), luego el resto de las duraciones son una fracci'on de 'esta.

Siendo que una nota no puede tener una duraci'on arbitraria, y esta relacionada con la duraci'on del beat, existen s'imbolos y nombres para diferentes duraci'ones. 
En la figura \ref{fig:durations} se muestran las figuras correspondientes a notas y silencios de distintas duraciones. De izquierda a derecha, las duraciones son las siguientes: A continuaci'on se exhibe una tabla con diferentes s'imbolos y duraciones.

\begin{imagen}
    \file{images/figures.png}
    \labelname{fig:durations}
    \desc{Figuras de ejemplo}
    \width{12cm}
\end{imagen}

Las partituras luego se organizan en una serie de unidades denominadas \emph{comp'ases}. Todos los compases tienen la misma duraci'on, y esta se especifica
en la partitura por una fracci'on al comienzo de la misma. 'Esta fracci'on no debe interpretar de forma usual, ya que posee una sem'antica sutilmente diferente.
En la especificaci'on del comp'as, el denominador determina en que unidades se medir'a la longitud del comp'as, y el numerador determina cuantas unidades dura'ra.
La figura \ref{fig:time_signatures} muestra los compases m'as comunes. 

\begin{imagen}
    \file{images/Common_time_signatures.png}
    \labelname{fig:time_signaturess}
    \desc{Un comp'as que ejemplifica la no inyectividad de la funci'on $c$}
    \width{5cm}
\end{imagen}

Es importante notar a fines de notar la duraci'on del compa'as, no existe diferencia entre el comp'as de $\frac{4}{4}$ y el comp'as de $\frac{2}{2}$.

numerador de se expresa a trav'es de una fracci'on donde el numerador y el denominador 
cumplen roles diferentes: El denominador nota la unidad de referencia, y el numera en valores relativos

La organizaci
temporal refiere a las duraci'ones de los distintos eventos que ocurren en una pieza musical. De esta forma, parte del lenguaje
musical esta dedicado a referirse a esta dimensi'on, que suele entenderse como horizontal. La dimensi'on altura responde a \emph{qu'e} suena, 
mientrasse refiere a la 
percepci\'on de las distintas notas que ocurren en un tema. Es importante notar la diferencia entre la altura de una nota y la frecuencia de la misma; la altura refiere
a la percepci'on de una frecuencia.

Si bien estas dos dimensiones se presentan por separado, de ninguna manera son independientes. Ser'an abordadas de esta forma 
para facilitar una primera descripci'on. De ser necesario, posteriormente se elaborar'an los conceptos que refieran a las interacciones entre 
estas dos componentes. 


z\subsection{Sobre la organizaci\'on altur\'istica \alert{!!}}
