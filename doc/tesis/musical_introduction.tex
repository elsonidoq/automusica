\section{Notas, alturas, acordes y tonalidades}
\label{sec:musical_intro}
Esta secci'on cumple dos objetivos dentro de este trabajo; por un lado, sirve para el lector no familiarizado con teor'ia musical para establecer
ciertos conceptos generales que son de gran importancia en el desarrollo de este trabajo. Por otro lado, permitir'a adem'as acentar un glosario
que se utilizar'a a lo largo de las diferentes secciones.

Principalmente, la m'usica tonal se divide en dos grandes componentes u ejes: la del \texttt{tiempo} y la de la \texttt{altura}. 

La dimensi'on temporal responde a \emph{cu'ando} ocurre un evento, mientras que la altura expresa caracter'isticas sobre \emph{qu'e}
tipo de evento es.  Dejando fuera del an'alisis a los instrumentos de percusi'on, la altura es refiere a la percepci'on de un cierto sonido. 
Esta percepci'on si bien esta relacionada con caracter'isticas intr'insecas del sonido, como su intensidad, timbre y frecuencia, 
tambi'en hace referencia
al proceso cognitivo que ocurre por parte del oyente. De esta forma, si bien un sonido de una frecuencia de $440Hz$ es percibido como una nota 
$La$, tambi'en lo es un sonido de frecuencia $441Hz$.

Dentro de la m'usica occidental, existen criterios dentro de ambas componentes que ayudan a la organizaci'on de una pieza musical. 
En lo subsiguiente se introducir'an ciertos conceptos de la teor'ia musical, puesto que son necesarios
para abordar los fundamentos cognitivos y los modelos propuestos. 

Por 'ultimo, dado que a lo largo del presente trabajo se exhiben ejemplos, es necesario que el lector sepa comprender una partitura sencilla, de forma que se se 
explicar'a brevemente como ciertos conceptos introducidos se notan en el lenguaje musical.

\subsection{Sobre la organizaci\'on temporal}
El primer concepto importante a tener en cuenta, es que el tiempo es \emph{discreto}. Esto no quiere decir que a la hora de la producci'on musical lo sea, sino que 
en la abstracci'on musical, existe una unidad de referencia temporal, a la que se llama \emph{beat}. En torno a ella se organiza la estructura temporal, tocando
siempre notas que duren una fracci'on del mismo. De esta forma, para poder reproducir una partitura es necesario saber la duraci'on de un \emph{beat} (denominado
\emph{tempo}), luego el resto de las duraciones son una fracci'on de 'esta.


En su libro, \citet*{LerdahlJackendoff83} hacen una distinci'on entre dos estructuras que ocurren en simultaneidad en la m'usica tonal:
La estructura \emph{m'etrica} y la estructura del \emph{agrupamiento}\footnote{\emph{meter} y \emph{grouping} en ingl'es}. 
Esta distinci'on en t'erminos generales coincide con clasificaciones de otros autores (\alert{citas muchas!}), se toma la version de Lerdahl y Jackendoff 
por ser esta m'as computacional.
La estructura del agrupamiento hace referencia a la organizaci'on de una pieza musical en unidades que pueden ser motivos, frases, secciones, etc. 
Cada una de estas unidades, es denominada por los autores como \emph{grupo}. Asimismo, el oyente infiere una estructura regular de beats. 
Algunos beats reciben una acentuaci'on mayor que otros, determinando lo que los autores definen como la estructura m'etrica. Dentro de esta estructura
existen distintos niveles, cada uno determinado por el tiempo transcurrido entre dos beats consecutivos. 
Un nivel posible es el llamado \emph{tactus} que corresponde de alguna forma al \emph{beat principal del tema}\footnote{Cognition of basic musical structures, p'agina 52}.
Se referir'a por contexto m'etrico de una pieza a la estructura m'etrica que se infiere a partir de esta.

Ambas estructuras son jer'arquicas, en el sentido de que existen sub-estructuras a distintos niveles, y que las sub-estructuras de niveles superiores 
incluyen por completo a las de nivel inferior
\footnote{Para una definici'on m'as formal de la jerarqu'ia referirse a \cite[p. XXX]{LerdahlJackendoff83}}. De esta forma, una secci'on
de una pieza musical estar'a formada por una sucesi'on de frases. Estas frases s'olo pertenecer'an a esa secci'on, sin embargo, esto no quiere decir
que no se pueda repetir una frase en dos secciones distitnas, puesto que cada una pertenecer'a a una sola secci'on. 

%\red{este parrafo no est'a bien, lo dejo para acordarme de reescribirlo}
%
%La organizaci'on jer'arquica de la m'usica no es una teor'ia solo postulada por Lerdahl y Jackendoff; Cooper y Meyer (\cite{CooperMeyer60}) plantean una teor'ia similar en
%estos t'erminos, Kramer (\cite{Kramer88}) si bien no habla directamente de una jerarqu'ia, propone que tienen un rol 
%estructurante\alert{tengo que leer m'as, as'i se mejor de lo que hablan estos dos muchachos}. 
%
%Un nivel de especial inter'es, es el denominado \emph{tactus}, que b'asicamente es el marcado por el director de orquesta al mover su batuta. 
%El tactus tambi'en es la distancia entre los beats que el oyente marca cuando mueve el pie y est'a relacionado con el baile. 
%
Es importante notar que esta estructura es ambigua, en el sentido de que que muchas veces no existe un 'unico an'alisis de una pieza musical 
en una estructura m'etrica y de agrupamiento.


Retomando con la duraci'on de las notas, dado que cada nota no puede tener una duraci'on arbitraria, y que esta duraci'on est'a relacionada con el tempo
a trav'es de una fracci'on, existen s'imbolos y nombres para diferentes duraci'ones. 
En la figura \ref{fig:durations} se muestran las figuras correspondientes a notas y silencios de distintas duraciones. Se denomina plica, a la barra vertical que surje 
de la cabeza de las notas, con excepcion de la redonda. Es importante aclarar que cuando dos notas de duracion inferior o igual a una corchea se tocan juntas, 
se unen las plicas, m'as adelante se dar'a un ejemplo.

\begin{imagen}
    \file{images/figures.png}
    \labelname{fig:durations}
    \desc{Figuras de ejemplo}
    \width{12cm}
\end{imagen}

Las partituras luego se organizan en una serie de unidades, cada una denominada \emph{comp'as}. Todos los compases tienen la misma duraci'on, y esta se especifica
en la partitura por una fracci'on al comienzo de la misma. Esta fracci'on no debe interpretar de forma usual, ya que posee una sem'antica sutilmente diferente.
En la especificaci'on del comp'as, el denominador determina en que unidades se medir'a la longitud del comp'as, y el numerador determina cuantas unidades 
este durar'a. El valor de referencia para el denominador es el $4$ refiriendo a una negra, y m'ultiplos de este refieren a m'ultiplos de una negra. Por ejemplo
$8$ refiere a corcheas.  De esta forma, un comp'as de $\frac{3}{4}$ indica que la unidad de referencia es la negra, y que hay $3$ negras por comp'as.
La figura \ref{fig:time_signatures} muestra los compases m'as comunes. 

\begin{imagen}
    \file{images/Common_time_signatures.png}
    \labelname{fig:time_signatures}
    \desc{Compases comunes}
    \width{5cm}
\end{imagen}

Observar que a fines de notar la duraci'on del comp'as, no existe diferencia entre el comp'as de $\frac{3}{4}$ y el comp'as de $\frac{6}{8}$, su diferencia
radica en la acentuaci'on recibida por distintos eventos: En el comp'as de $\frac{3}{4}$, existen 3 eventos de negra, donde el primero se percibe como acentuado, 
y los dos siguientes se perciben con una acentuaci'on m'as d'ebil, mientras que en el comp'as de $\frac{6}{8}$ hay 6 eventos, donde el primero es el m'as 
acentuado, luego sigue el cuarto, y por ultimo el resto de los beats.

Adem'as de los s'imbolos explicados, existen ciertos s'imbolos que permiten que el lenguaje sea m'as rico, y describir nuevas duraciones. La ligadura
permite sumar la duraci'on de dos figuras. Un caso particular de la ligadura, es el puntillo. El puntillo se nota como un peque~no punto que se coloca a 
continuaci'on de la cabeza de una nota, y determina que la duraci'on de esa nota debe multiplicarse por $\frac{3}{2}$. 
De esta forma, una \emph{negra con puntillo}, dura una negra y media. 

Para fijar conceptos, se exhibe una partitura de ejemplo en la figura \ref{fig:example_measure}. 
La duraci'on del primer s'imbolo es $\frac{3}{4}=\frac{1}{2}\times\frac{3}{2}$,
la duraci'on del segundo s'imbolo es $\frac{1}{4}$, puesto que tiene dos plicas, y eso indica que es una semicorchea. Luego sigue un silencio de negra,
y una negra ligada a una semi corchea, es decir, su duracion sera $1 + \frac{1}{4}$, luego sigue una semi corchea y un silencio de corchea, completando
con la duraci'on total de 4 negras. La letra $c$ al principio del comp'as es una abreviaci'on para decir $\frac{4}{4}$.

\begin{imagen}
    \file{images/pedagogic_measure.png}
    \labelname{fig:example_measure}
    \desc{Ejemplo para fijar conceptos}
    \width{12.5cm}
\end{imagen}

\subsection{Sobre la organizaci\'on de las alturas}
La m'usica tonal refiere a m'usica donde una altura funciona como punto de referencia. Este punto de referencia es una nota llamada t'onica y es
la que le da nombre a la escala; por ejemplo Do es la t'onica de la escala de Do mayor. 

Existen dos formas alternativas de denominar a las notas, la usual que utiliza los nombres Do, Re, Mi, Fa, Sol, La y Si y la americana, que 
asigna letras a cada nota resultando en C, D, F, E, G, A y B respectivamente. Se utilizar'an ambas formas para referirse a las notas de forma
intercambiable. 

Continuando con las escalas, las escalas mayores y menores de la m'usica occidental, llamadas escalas diat'onicas, 
son un subconjunto de siete alturas de la escala crom'atica.
La escala crom'atica es aquella que divide, sobre una escala logar'itmica, en 12 partes iguales a las frecuencias cuya raz'on es 2. Por ejemplo, 
la frecuencia de la nota $A$ es $440Hz$. En la tabla \ref{tab:cromatica} se exhibe las frecuencias de las notas de la escala crom'atica 
comenzando desde $A$.

\begin{table}[h]
\begin{center}
\begin{tabular}[c]{|l|c|}
\hline
\textbf{Nota} & \textbf{Frecuencia en $Hz$} \\
\hline 
A		&	440.00 \\
A\#		&	466.16 \\
B		&	493.88 \\
C		&	523.25 \\
C\#		&	554.37 \\
D		&	587.33 \\
D\#		&	622.25 \\
E		&	659.26 \\
F		&	698.46 \\
F\#		&	739.99 \\
G		&	783.99 \\
G\#		&	830.61 \\
A		&	880.00 \\ 
\hline
\end{tabular}
\label{tab:cromatica} 
\end{center}
\end{table}


Utilizando la escala crom'atica se puede definir la noci'on de \emph{semitono} y de \emph{intervalo mel'odico}. Un semitono es la distancia
entre dos notas consecutivas dentro de la escala crom'atica. Por ejemplo $E$ y $F$ est'an a un semitono de distancia. Un intervalo mel'odico, 
es la distancia entre dos notas medida en semitonos .Dentro de la teor'ia musical, a estos intervalos se les pone un nombre que permite \red{>qu'e permite?}. 
Como a lo largo de este trabajo no se utilizaran los nombres de los intervalos mel'odicos, se los nombrar'a como la cantidad de semitonos que separa
a las notas que lo conforman. Por ejemplo, las notas $A$ y $B$ est'an a 2 semitonos de distancia, por lo que el intervalo mel'odico
se notar'a como \IM{2}. 

Como se mencion'o al principio de esta secci'on, de la escala crom'atica se elije un subconjunto de 7 notas. Estas notas ser'an referenciadas
respecto de una nota normativa. El tipo de escala con la que se est'e trabajando depender'a del patr'on de intervalos que forman las 7 notas elegidas
respecto de la t'onica. Por ejemplo, la escala mayor forma el siguiente patr'on:

\begin{center}
\IM{2} \IM{4} \IM{5} \IM{7} \IM{9} \IM{11} \IM{12}
\end{center}

Si se observa el intervalo que forma cada nota respecto de la anterior, el patr'on resultante es el siguiente

\begin{center}
\IM{2} \IM{2} \IM{1} \IM{2} \IM{2} \IM{2} \IM{1}
\end{center}

Si se rota este patr'on un lugar hacia la izquierda, se obtiene el patr'on que forman las escalas menores:

\begin{center}
\IM{2} \IM{1} \IM{2} \IM{2} \IM{2} \IM{1} \IM{2} 
\end{center}

De esta forma, la escala mayor de C estar'a formada por las notas C, D, E, F, G, A, B, mientras que la escala menor de C estar'a formada
por las notas C, D, D\#, F, G, G\#, A\#, B\footnote{Por cuestiones de claridad se escribe la escala de esta forma. La forma correcta
de escribirlo es C, D, Eb, F, G, Ab, Bb, ya que cada nota debe aparecer solo una vez. Se entiende que este detalle es irrelevante a los efectos
de lo que se desea transmitir}


Por 'ultimo, la entidad restante a hacer mensi'on es el \emph{acorde}. Un acorde est'a formado por tres o m'as notas. Cada acorde abarca una regi'on
en el tema, y dentro de la regi'on de dominancia ser'a el foco arm'onico.

Tornando el foco a la lectura musical, existen dos elementos que permiten notar de significado las l'ineas y espacios del pentagrama: la clave
y la armadura de clave. La clave asigna una nota a un rengl'on en particular, y la armadura permite especificar en que escala se tocar'a. 
Dado que a lo largo de este trabajo solo se dar'an ejemplos sobre la tonalidad mayor, se obviar'a la explicaci'on de que es la armadura de clave. 
La clave utilizada para todos los ejemplos de esta tesis ser'a la clave de sol que determina que la segunda l'inea, contando de abajo hacia
arriba, ser'a la nota G. El espacio entre esta l'inea y la siguiente, ser'a la nota A, y as'i sucesivamente. Si se deseara especificar
la nota A\#, basta con anteponer el s'imbolo \# a la cabeza de la nota. En la figura \label{fig:escala_mayor} se exibe la escala de C mayor.


\begin{imagen}
    \file{images/major_scale.png}
    \labelname{fig:escala_mayor}
    \desc{Escala mayor de C}
    \width{12cm}
\end{imagen}
