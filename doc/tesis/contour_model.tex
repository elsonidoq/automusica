\section{Modelando contornos mel\'odicos}
\label{sec:melodic_contour}
Las caracter'isticas definidas en la secci'on anterior son en gran parte restricciones que deber'ian tenerse en cuenta en la construcci'on 
de l'ineas mel'odicas respecto a una cierta pieza musical, sin embargo, ninguna de ellas habla de propiedades de la l'inea mel'odica 
por si s'ola.

En 1960, Leonard Meyer elabor'o una teor'ia acerca de la expectativa en la m'usica aplicando principios gest'alticos de la psicolog'ia. La 
psicolog'ia de la Gestalt define principios que pretenden capturar la forma en que la mente configura los elementos que llegan a ella a trav'es 
de la percepci'on o de la memoria. Por ejemplo, la ley de cierre establece que nuestra mente a~nade los elementos faltantes a para completar una 
figura. De esta forma, en la figura \ref{fig:ley_cierre} se puede ver un c'irculo y un rectangulo, si bien en la imagen s'olo hay partes.

\begin{imagen}
    \file{images/Gestalt_ley_de_cierre.png}
    \labelname{fig:ley_cierre}
    \desc{Ejemplo de la ley de cierre. Si bien en esta imagen no hay un circulo ni un rectangulo, nuestra mente lo completa. }
    \width{6cm}
\end{imagen}

Luego del trabajo de Meyer, Eugene Narmour en (\cita) cuantifica estas reglas en t'erminos de intervalos para construir una teor'ia de 
la expectativa de los contornos mel'odicos. 

En lo que sigue, se explican con mayor detalle las teor'ias de Narmour, para luego detallar el modelo de las l'ineas mel'odicas.

\footnote{definir pitch class y nota y equivalencia entre notas de diferente octava en las secciones de background} 

\subsection{La teor\'ia de la Implicaci\'on-Realizaci'on}
Como se mencion'o en la introducci'on de este cap'itulo, Eugene Narmour, basado en la teor'ia de la expectativa musical de Leonard Meyer, propuso una forma para cuantificar
el grado de expectativa sobre el \emph{contorno melod'ico}. El contorno mel'odico est'a conformado de la sucesi'on de intervalos que ocurren en una melod'ia. Por ejemplo, 
en la figura \ref{fig:simple_melody} se exhibe una melod'ia en donde se tocan las notas Do, Re, Do, Fa\#. 

\begin{imagen}
    \file{images/melody.png}
    \labelname{fig:simple_melody}
    \desc{Melod'ia simple de ejemplo}
    \width{11cm}
\end{imagen}

El contorno mel'odico de la figura \ref{fig:simple_melody} ser'a entonces \IM{2}, \IM{-2}, \IM{6}.
Notar que esta transformaci'on no es biyectiva, puesto que la melod'ia Re, Mi, Re, Sol\# tiene el mismo contorno.

\alert{este parrafo esta enterito sujeto a revision =D}

El modelo de la Implicaci'on-Realizaci'on(I-R) toma de (\cita) que el sistema cognitivo se encuentra organizado jer'arquicamente. En esta jerarqu'ia, los sistemas 
de percepci'on m'as simples, como la vista, se encuentran al fondo y los sistemas de percepci'on mas abstractos o complejos como la memoria o la 
resoluci'on de problemas se encuentran en el tope. De esta forma se distinguen dos tipos de procesos expectaci'on denominados procesos \emph{bottom-up} 
y \emph{top-down}. Los procesos bottom-up son aquellos procesos cognitivos en donde se parte de una informaci'on ubicada en los niveles bajos de la jerarqu'ia y se la 
elabora llev'andola a los niveles altos, mientras que los procesos top-down lo hacen en el orden inverso. 

Narmour propone que los procesos que regulan la expectaci'on mel'odica son en mayor medida de tipo bottom-up, es decir, parten de informaci'on puramente sensorial. 
Carol Krumhansl resume en (\cita) que seg'un este modelo, la cognici'on de melod'ias puede ser descripta como una sucesi'on de puntos 
de \emph{cierre}, \emph{implicaci\'on} y \emph{realizaci\'on}. Cuando se alcanza un punto de cierre, las expectativas sobre la continuaci\'on 
son d'ebiles mientras que cuando se alcanza un punto de implicaci'on, la expectativa sobre la continuaci'on es fuerte. Seg'un este modelo,
existen seis condiciones que llevan a una sensaci'on de cierre:
\begin{itemize}
 \item Silencio
 \item Una acento metro fuerte
 \item Una disonancia resolviendo a una consonancia
 \item Una nota corta seguido de una larga
 \item Un intervalo grande seguido de un intervalo m'as peque~no
 \item Un cambio en la direcci'on registral
\end{itemize}

Estas caracter'isticas no son exluyentes, y si ocurren varias en simultaneo, la sensaci'on de cierre es mayor. 
Si no ocurre ninguna de esas condiciones, entonces se establece un punto de implicaci'on, y el 'ultimo itervalo mel'odico recibe el nombre
de \emph{intervalo implicativo}, y el intervalo que sigue al implicativo, recibe el nombre de \emph{intervalo realizados}. 
Un intervalo realizado, puede promover o no una sensacion de cierre, y puede satisfacer o no la implicaci'on establecida por el intervalo 
implicativo
 
Teniendo entonces el contorno de una melod'ia, Narmour define ciertas caracter'isticas relacionadas con la psicolog'ia de la gestalt. A continuaci'on 
se enumeran los cinco principios de esta teor'ia, refiri'endose por intervalos peque~nos, a intervalos de valor absoluto menor o igual a 5 semi tonos, y por intervalos
grandes a aquellos que sean mayores o iguales que 7 semitonos, dejando al intervalo de 6 semitonos fuera de la clasificaci'on.
\begin{enumerate}
 \item Direcci'on registral: intervalos peque~nos implican continuaci'ones en la misma direcci'on mel'odica, mientras que intervalos grandes implican un cambio de direcci'on
 \item Diferencia interv'alica: intervalos peque~nos implican otros de tama~no similar y que intervalos grandes implican intervalos m'as peque~nos. 
 \item Retorno registral: se cumple cuando la segunda nota del intervalo realizado es id'entica o similar a la primera del intervalo implicativo.
 \item Proximidad: el tama~no del intervalo relizado sera peque~no.
 \item Cierre: se cumple cuando hay un cambio de direcci'on, un movimiento hacia un intervalo m'as peque~no, o ambas situaciones a la vez (\alert{reescribir})
\end{enumerate}

\subsection{El modelo}
Como es de esperarse, no existe una 'unica forma de construir un modelo que refleje este comportamiento. A continuaci'on se proponen dos modelos. El primero 
parte de una simplificaci'on a algunos principios de la teor'ia de Narmour. Estas simplificaciones permiten construir una cadena de Markov tradicional para el contorno 
mel'odico. Se presenta este modelo como punto de comparaci'on con el segundo que, si bien sigue simplificando la teor'ia de Narmour, lo hace en menor medida. 

\subsection{Simplificaci\'ones a I-R}
Dada un pieza musical, en este caso es de inter'es construir un modelo y entrenarlo con el contorno mel'odico del tema en cuesti'on. De esta forma se puede proyectar
las alturas de las notas del tema, y luego construir la sucesi'on de intervalos mel'odicos tomando la diferencia entre las sucesivas alturas. Por ahora as'umase que 
en el tema de entrenamiento no hay notas sonando en simult'aneo, puesto que en ese caso, no es trivial construir una sucesi'on de intervalos mel'odicos que represente
lo que esta ocurriendo. Se tratar'a con esa cuesti'on m'as adelante.

La primera simplificaci'on a la teor'ia de Narmour, compartida por los dos modelos, es no analizar que intervalos las condiciones que hacen que un intervalo realizado
no sea implicativo, es decir, se asume que todos los intervalos son implicativos. Se considera que la desici'on de donde colocar intervalos de cierre es candidata 
a estar ubicada dentro a otro nivel en la jerarqu'ia de modelos para construir lineas mel'odicas, puesto que en general los cierres estan asociados a la estructura
del tema, de esta forma, para mantener limitado el alcance de esta tesis, se deja para trabajo a futuro trabajar con puntos de cierre.

Por otro lado, teniendo en mente el objetivo de construir de una cadena de Markov y en particular elegir el espacio de estados, se realiz'o una modificaci'on al enunciado 
del principio de diferencia interv'alica: En lugar de codificiar que intervalos peque~nos impliquen intervalos de tama~no \texttt{similar}, se codific'o que 
intervalos \texttt{peque~nos} implican intervalos \texttt{peque~nos}. Para los intervalos grandes se hizo lo mismo: intervalos \texttt{grandes} implican 
intervalos \texttt{grandes}. Esta simplificaci'on hace que a la hora de generar melod'ias, no sea necesario conocer el tama~no del intervalo recien generado, y baste 
con saber si 'este fue grande o peque~no.

\subsection{La cadena de Narmour}
Con las simplificaciones propuestas a continuaci'on se detalla el modelo de Markov para la teor'ia de Implicaci'on-Realizaci'on.
Dada una sucesi'on de alturas $a_1, \cdots, a_n$ se construye la sucesi'on de intervalos mel'odicos $m_1, \cdots, m_{n-1}$ donde $m_i = a_{i+1}-a_i$. 

Se define el predicado $T(m)$ que toma 3 valores: grande, peque~no o mediano utilizando la clasificaci'on de Narmour y definiendo mediano como el intervalo de 
6 semitonos que queda fuera de la clasificaci'on. Se define tambi'en el predicado $D(m)$ que toma dos valores: ascendente y descendente seg'un el signo del intervalo.

De esta forma, los estados de la cadena de Markov estar'an dados por vectores $<T, D>$ para los distintos valores de estos predicados. Notar que esta representaci'on
resulta en un modelo muy compacto con $6$ estados en total, lo cual permite entrenar con temas cortos.

De esta forma, la sem'antica de realizar una caminata en este modelo impone restricciones sobre el tipo de intervalo a tocar, dado que se sabe el tipo de intervalo que se 
toc'o previamente.

