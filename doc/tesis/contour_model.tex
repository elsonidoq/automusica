\section{Modelando l\'ineas mel\'odicas}
\comment{En esta secci'on me focalizo en la partitura como una sucesi'on de alturas}

\subsection{Contextos}
\comment{Aca la idea es explicar que hay que tener en cuenta dos tipos de contextos cuando se trabaja con la melod'ia, uno tiene que ver con cuestiones de las notas que vengo 
tocando, y el otro con el momento en el que estoy de la pieza (que acorde esta sonando). Basicamente, hay un contexto horizontal y otro vertical}

\subsection{La teor\'ia de la Implicaci\'on-Realizaci'on}
\comment{Aca explico mas o menos la teoria de narmour, y cuento como me puede servir para modelar el contexto horizontal}

\subsection{Contexto vertical}
\comment{Yo aca uso un modelo que defini a dedo y que esta todavia sujeto a cambio, me gustaria discutir esto con Favio a ver si esta de acuerdo con lo que estoy haciendo. De todas formas me gustar'ia armar una discuci'on aca. Como contexto vertical, tambien se puede tener en cuenta la tonalidad (en menor medida que el acorde). Me gustaria ver si krumnhansl en su libro cognitive fundations of musical pitch me podria dar algo sobre como llenar esta parte o como hacer un modelo mas interesante (no se si quiero hacer un modelo mas interesante de todas formas =p)}

\subsection{El modelo}

\subsubsection{La cadena de Narmour}
\comment{aca la idea es explicar la cadena de markov de los intervalos de narmour}

\subsubsection{Combinaciones convexas}
\comment{Aca explico como las combinaciones convexas de ciertas distribuciones de probabilidad me determinan el contexto armonico}

\subsection{El proceso generativo}
\comment{Aca tengo que explicar el problema de mezclar estos dos modelos no es trivial, porque se te podria trabar el proceso porque llega a un callejon sin salida, y muestro
como deber'ia ser el proceso compuesto posta}
