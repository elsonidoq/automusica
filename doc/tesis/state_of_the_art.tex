\section{Aportes de este trabajo}
Se considera como aporte principal de este trabajo la metodolog'ia de validaci'on propuesta: 

Contando con modelos generativos aislados para distintos atributos musicales, es posible analizar el impacto que tiene un cierto 
atributo en la cognici\'on musical
evaluando una serie de composiciones afectando el funcionamiento del modelo correspondiente.


Dado que este objetivo no ha sido planteado con anterioridad, es de esperarse que no se hayan construido modelos de estas caracter'isticas. Si bien existen excepciones, 
como el modelo para el contorno mel'odico planteado en \cite{PaieThesis}, esto no ocurre con la mayor'ia de los modelos de la literatura. 

Para poder alcanzar este objetivo, fue necesario construir modelos ge\-nerativos con respaldo cognitivo para los distintos atributos que caracterizan a una melod'ia.

De este modo, la contribución de este trabajo en cuanto a tales necesidades de modelización consiste en:

\begin{itemize}
 \item Modelo para la r\'itmica: Basado en 3 propiedades perceptuales del acento m'etrico, fue posible construir un modelo sencillo para la r'itmica
 que cumpla con la propiedad de que la r'itmica generada por este modelo har'a que un oyente infiera la misma estructura m'etrica que la que el oyente inferir'ia 
 en la obra original.

 \item Modelo para los contornos mel'odicos: Basado en la teor'ia de Implicaci'on-Realizaci'on de \cite{Narmour91}, se propone un modelo simple que permite cuantificar 
 el grado de implicaci'on que genera un intervalo mel'odico sobre el siguiente.

 \item Modelo para contextos arm'onicos: Basado en los trabajos de \cite{Krumhansl90} y \cite{Lerdahl2001} se propone un modelo basado en un framework bayesiano
 para cuantificar la estabilidad de una nota en funci'on de la tonalidad inferida del tema y el acorde que gobierna la sonoridad del momento.

 \item Modelo para frases mel'odicas: Utilizando los modelos anteriores, y basado en algunas definici'ones de en qu'e consiste una frase musical, se construye un modelo
 que respete los anteriores y que adem'as cumpla ciertas propiedades que respectan a una frase musical.

 \item Modelos para generar motivos: Se propone utilizar el modelo de los Restaurantes Chinos \citep{Teh2007} para generar repetici'on de partes, 
 de forma tal que la cantidad de repeticiones de una parte no sea excesiva ni nula.

\end{itemize}

El resto de la tesis se organiza de la siguiente forma: En el cap'itulo 2 se cubren superficialmente los conceptos b'asicos que el lector debe manejar para entender
este trabajo. Se alienta a los lectores no iniciados en alguna de las tem'aticas abordadas profundizar tales conceptos en la literatura referida. 
En el cap'itulo 3 se exhibe modelos que capturan dependencias a nivel local en la m'usica tonal. 
En el cap'itulo 4 exhibe modelos de 'indole jer'arquico y por 'ultimo en el cap'itulo 5 se exhiben los experimentos realizados junto con 
las conclusiones del trabajo.

A modo resumen, en la figura \ref{fig:arquitectura} se exhibe la arquitectura de los modelos utilizados y en qu'e cap'itulo se aborda cada uno.

\begin{imagen}
    \file{images/arq.png}
    \labelname{fig:arquitectura}
    \desc{Descripci'on gr'afica de la arquitectura utilizada}
    \width{10.5cm}
    \position{!h}
\end{imagen}

\begin{itemize}
 \item Pitch Profile: Construye el pitch profile definido en la secci'on \ref{sec:pitch_profile}.
 \item Chord Detection: Aplica la heur'istica de detecci'on de acordes.
 \item Contour Patterns: Calcula el modelo definido en la secci'on \ref{sec:contour_model}.
 \item Notes Distribution: A partir del Pitch Profile y del acorde actual construye la distribuci'on de notas que aplica al contexto actual seg'un se defini'o en la secci'on \ref{sec:harmonic_context_model}.
 \item Phrase Repetition: Utilizando el Restaurant Chino correspondiente al acorde actual, se elije un identificador para la parte actual como 
 se defini'o en la secci'on \ref{sec:crp_model}.
 \item Phrase Rhythm: Utilizando la duraci'on determinada por el acorde actual y el identificador determinado por la etapa de Phrase Repetition se 
 genera una frase r'itmica como se defini'o en la secci'on \ref{sec:rythm_model}.
 \item Pitch Phrase: Utilizando la r'itmica generada por la etapa de Phrase Rhythm, se llenan los \emph{slots} que esta determina, utilizando
 como contexto arm'onico el determinado por Notes Distribution y utilizando como restricci'ones para el contorno mel'odico las determinadas por
 Contour Patterns utilizando el algoritmo definido en la secci'on \ref{sec:phrase_model}.
\end{itemize}


\section{Estado del arte}
\label{sec:state_art}
A continuaci'on se har'a un breve resumen sobre los distintos enfoques y objetivos relacionados con el an'alisis musical mediante
t'ecnicas computacionales. Las aplicaciones de este tipo de trabajos son bastante variadas, tales como herramientas para 
composici'on musical asistida, acompa~namiento autom'atico, indexaci'on de m'usica, sistemas de recomendaci'on musical e 
investigaci'on en psicolog'ia de la m'usica, donde se situa este trabajo.

Seg'un cu'al sea el enfoque que se tome, podr'a variar la representaci'on musical utilizada entre la se~nal sonora cruda, y una partitura o midi.
En este trabajo se asumir'a una representaci'on simb'olica de la m'usica para poder desarrollar en profundidad las teor'ias cognitivas de las 
que luego se hablar'a. A continuaci'on se nombran algunos trabajos que se enmarcan de la misma forma en t'erminos de la representaci'on musical.

\cite{PaieThesis} es el trabajo m'as cercano a 'este. En 'el se propone un modelo generativo para l'ineas mel'odicas, 
patrones r'itmicos y armonizaci'ones, basado en algunos de los principios que se utilizar'an en este trabajo. 
Sin embargo, no es el objetivo del autor poner a prueba teor'ias cognitivas de la m'usica; sino desarrollar un modelo generativo,
y por lo tanto predictivo, basado en una representaci'on simb'olica de la m'usica de forma tal que se pueda utilizar el conocimiento generado por estos
modelos para mejorar la calidad de algoritmos que trabajan con se~nales sonoras. M'as all'a de que no se comparten los objetivos entre el trabajo de 
Paiement y este, no es posible tampoco utilizar los modelos propuestos por 'el puesto que hay teor'ias que no utiliza, como la de \cite{Lerdahl2001} 
y varios de ellos, como el de la r'itmica, no tienen una fundamentaci'on cognitiva clara.

\cite{Shih-Chuan} propone un software que mediante la utilizaci'on de t'ecnicas de data mining componga m'usica. Nuevamente, estos modelos
no son de inter'es para este trabajo puesto que pr'acticamente no tienen ning'un fundamento cognitivo, y siendo as'i, no podr'a realizarse 
ning'un tipo de validaci'on.


Otro software existente dise~nado para componer m'usica es el Melisma Music Generator
\footnote{Disponible en http://www.link.cs.cmu.edu/melody-generator/} basado en \cite{Temperley2004}. 
Este software se encuentra disponible para escuchar online sus composiciones, sin embargo, el mayor problema 
que tiene es que no se puede entrenar directamente su modelo con una partitura, y por m'as que se intentara realizar esto construyendo un software 
que estime valores posibles para sus par'ametros, no habr'ia forma de determinarle una sucesi'on de contextos arm'onicos.


%David Cope, en trabaja con un software, que mediante reglas, sea capaz de reproducir el estilo de una pieza musical dada. Si bien 
%la construcci'on ha sido exitosa en algunos casos, generando composiciones que realmente respetan el estilo de la pieza original, gran
%parte de las reglas utilizadas no tienen sustento cognitivo.
%

\cite{Cambouropoulos98} propone una teor'ia general para el an'alisis musical. Durante el desarrollo de esta tesis no se tuvo conocimiento de la existencia del trabajo de Cambouropoulos.
Luego de leer parcialmente su trabajo, se encontr'o que si bien sus objetivos son distintos a los que se proponen aqu'i, no distan tanto: 
\begin{quote}
[\ldots]Musical theories allow the formulation of hypotheses and models which can be implemented
as computer programs and then evaluated, and, conversely, results from the application of the
computer programs may force the re-examination and adjustment of the initial theories. [\ldots]
\end{quote} 
Si bien el objetivo de su trabajo es construir una teor'ia descriptiva, es posible que el an'alisis que su teor'ia provee hubiera permitido llegar a modelos m'as precisos 
en algunos casos. 

%
%En \cite{Simon_mysong:automatic} mediante el an'alisis de caracter'isticas de la se~nal sonora correspondientes a una melod'ia 
%cantada, y mediante entrenar un Hidden Markov Model para la probabilidad de un acorde, dado que se observa una cierta nota cantada,
%se construy'o un sistema capaz de armonizar una melod'ia. Un sistema comercial que realiza esto mismo es Band-in-a-box,
%pero dado que no existen \red{revisar} publicaciones respecto a c'omo fue construido, no se puede hacer m'as que nombrarlo.
%
%Dentro de la rama de sistemas para indexar m'usica se situan trabajos como \cite{StructureAnalysis1}. En 
%\cite{StructureAnalysis1} se propone un m'etodo para analizar la estructura de una pieza musical a partir de su se~nal sonora. Esto lo logran extrayendo vectores 12-dimencionales
%de cada momento del tema, en donde cada componente del vector muestra la intencidad relativa de cada altura en ese momento, y a partir
%de estos vectores y una noci'on de distancia construyen matrices de similitud que permiten detectar los acordes que aparecen
%en el tema. Siguiendo con este 

