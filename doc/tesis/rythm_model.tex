\section{Modelando la m\'etrica}
 \comment{retomar la importancia de los acentos metricos contada en la seccion \ref{sec_cogn_bg} y citar el articulo que dice que el tiempo se 
 puede partir en clases de equivalencia por los acentos metricos. }\newline
El acento m'etrico es una caracter'istica inherente a la m'usica tonal; cualquier pieza musical se enmarca en una estructura 
(posiblemente ambigua) de beats fuertes y d'ebiles. Esta estructura permite medir el tiempo permitiendo a un interprete reproducir y a un oyente
reconocer un cierto conjunto de relaci'ones temporales. La estructura m'etrica contribuye a la organizaci'on r'itmica de una pieza musical, 
refiri'endose a la r'itmica como un grupo de al menos dos eventos musicales en donde hay uno acentuado con respecto al resto \cite{CooperMeyer60}.
Un patr'on temporal as'i descripto es interpretado de forma distinta seg'un el contexto m'etrico donde sea escuchado. 

 \comment{Mencionar la ``localidad'' de los acentos metricos (no se perciben a niveles altos)}\newline
Siguiendo la teor\'ia de Lerdahl y Jackendoff, el acento m'etrico es una construcci'on mental, que si bien es jer'arquica, lo niveles altos no son percibidos, puesto
que a esos niveles entran en juego los acentos estructurales y los acentos fenom'enicos. De esta forma se habla de la \emph{localidad} de los acentos m'etricos: s'olo se perciben a 
niveles bajos en la jerarqu'ia\footnote{meter en la intro la nocion de nivel alto $\leftrightarrow$ timespan largo}. 

\comment{Mencionar la periodicidad como parte de la definicion de acento metrico que utilizo}


 \comment{Definir el paso del tiempo como saltos entre las clases de equivalencia}\newline
La estructura m'etrica permite organizar los distintos puntos en el tiempo en clases de 
equivalencia de forma tal que dos puntos distintos en el tiempo que pertenezcan a la misma clase de equivalencia ser'an percibidos de forma similar\cite{Benjamin84}. 
De esta forma, es posible pensar al paso del tiempo como saltos entre estas clases de equivalencia. 

La propiedad de localidad, junto con la propiedad de periodicidad, indica que para modelar este tipo de acentos, no es necesario un modelo que capture dependencias 
por ensima del ``punto de localidad``. 


 \comment{Concluir que pensando al paso del tiempo como saltos entre clases de equivalencias, y la localidad del acento metrico hace que sea
 mas o menos natural pensar una suceci'on de duraciones sea cocientadando por el periodo del ciclo\ldots el unico problema es inferir ese periodo}

 \comment{Primer aprouch: tratar de inferir el acento metrico y construir un algoritmo que dado un mapping de acentos metricos encuentre el periodo,
       Segundo aprouch: usar el compas.
       Mencionar problemas de ambas aproximaciones}

 \comment{hasta ahora podemos traducir de una partitura a una sucesion de clases de equivalencia, se puede tratar de aprender esto como un lenguaje,
 explicar el modelo de markov}

 \comment{formalizar y poner dibujitos}

 \comment{explicar el proceso generativo}

 \comment{relacionar esto con lo de contexto horizontal definido en \ref{subsec_tension}}


%
%\subsection{El modelo}
%
%
%
%
%
%
%
%
%
%
%
%
%
%
%
%El prop'osito de este modelo es permitir trabajar con la componente temporal de la m'usica por separado del resto, 
%de modo que s'olo se trabajar'a con la relaci'on entre las duraciones de las distintas notas que
%aparecen en la pieza de entrenamiento, quedando fuera del modelo tanto la altura como la falta de sonido (silencio).
%
%
%    
%
%
%La primer relaci'on, o al menos la m'as directa, entre la duraci'on de las notas y la escucha musical es el ac'ento 
%m'etrico. La estructura m'etrica es una caracter'istica importante dentro de la musica tonal, y ser'ia deseable que el modelo construido, de alguna forma la 
%respete. 
%Supongamos por un momento que es posible realizar un algoritmo que permita inferir el acento m'etrico de cada mom Esta caracterizaci'on del tiempo permitir'ia 
